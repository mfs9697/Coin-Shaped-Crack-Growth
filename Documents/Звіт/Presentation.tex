% !TeX program = lualatex
\documentclass[aspectratio=169]{beamer}

% -------------------------------------------------
% Мова та шрифти
% -------------------------------------------------
\usepackage{fontspec}
\usepackage{polyglossia}
\setmainlanguage{ukrainian}

\setmainfont{Libertinus Serif}
\setsansfont{Libertinus Sans}
\setmonofont{Libertinus Mono}

% -------------------------------------------------
% Оформлення Beamer
% -------------------------------------------------
\usetheme{Madrid}
\usecolortheme{default}
\setbeamertemplate{navigation symbols}{}

\setbeamertemplate{footline}
{
	\leavevmode%
	\hbox{%
		\begin{beamercolorbox}[wd=.9\paperwidth,ht=2.5ex,dp=1ex,leftskip=1em]{author in head/foot}%
			\usebeamerfont{author in head/foot}\insertshortauthor
		\end{beamercolorbox}%
		\begin{beamercolorbox}[wd=.1\paperwidth,ht=2.5ex,dp=1ex,center]{date in head/foot}%
			\usebeamerfont{date in head/foot}\insertframenumber
	\end{beamercolorbox}}%
}

% -------------------------------------------------
% Загальні пакети
% -------------------------------------------------
\usepackage{graphicx}
\usepackage{booktabs}
\usepackage{subcaption}
\usepackage{multicol}


% -------------------------------------------------
% Дані титулу
% -------------------------------------------------
\title{Науково-дослідна робота 1.3.1.436-22: 
	\\
	Розв’язання плоских та просторових задач механіки руйнування для тріщин нормального відриву із когезійними зонами 
}

\subtitle{Керівник:	професор, доктор фізико-математичних наук \\	
	А.О. Камінський }

\author[  Інститут механіки  ім.~С.П.~Тимошенка НАН України]{%
	Відділ механіки руйнування матеріалів \\
	Інститут механіки ім.~С.П.~Тимошенка\\
	Національної академії наук України
}

\institute{}

\date{2025}



% -------------------------------------------------
\begin{document}
	
	% =================================================
	% Титульний слайд
	% =================================================
	\begin{frame}
		\titlepage
	\end{frame}
	
	% =================================================
	% Етапи виконання роботи
	% =================================================
	\begin{frame}{Етапи виконання роботи}
		\begin{itemize}
			\item \textbf{I.2022--IV.2022.}
			Розробка методу розв’язання задач механіки руйнування
			для нелінійно-пружних тіл з тріщинами нормального відриву
			при двовісному навантаженні.
			
			\item \textbf{I.2023--IV.2023.}
			Розробка числово-аналітичних методів для розв’язання
			плоских і просторових задач механіки руйнування
			в’язкопружних елементів конструкцій, послаблених
			тріщинами з когезійними зонами.
			
			\item \textbf{I.2024--IV.2024.}
			Розв’язання нових плоских і просторових задач механіки
			довготривалого руйнування в’язкопружних тіл з тріщинами
			нормального відриву та визначення параметрів
			тріщиностійкості елементів конструкцій з полімерів
			та композитів.
			
			\item \textbf{I.2022--IV.2025.}
			Аналіз отриманих закономірностей руйнування
			нелінійно-пружних та в’язкопружних тіл і розробка
			рекомендацій щодо визначення тріщиностійкості та
			довговічності сучасних елементів конструкцій.
		\end{itemize}
	\end{frame}
	
	
	% =================================================
	% Виконавці роботи
	% =================================================
	\begin{frame}{Виконавці роботи}
		\small
		\begin{multicols}{2}
			\begin{itemize}
				\item \textbf{А.\,О.~Камінський}  
				\\ керівник НДР, проф., д.~ф.-м.~н.
				
				\item \textbf{М.\,Ф.~Селіванов}  
				\\ в.~о. зав. відділом,  
				чл.-кор. НАН України, д.~ф.-м.~н.
				
				\item \textbf{Р.\,М.~Мартиняк}  
				\\ пров. н.с., д.~ф.-м.~н.
				
				\item \textbf{О.\,С.~Богданова}  
				\\ с.н.с., к.~ф.-м.~н.
				
				\item \textbf{Є.\,Є.~Курчаков}  
				\\ с.н.с., к.~ф.-м.~н.
				
				\item \textbf{Ю.\,О.~Чорноіван}  
				\\ с.н.с., к.~ф.-м.~н.
				
				\item \textbf{Л.\,М.~Терещенко}  
				\\ с.н.с., к.~ф.-м.~н.
				
				\item \textbf{С.\,М.~Масалова}  
				\\ ст. технік
			\end{itemize}
		\end{multicols}
	\end{frame}
	
	
	% =================================================
	% Розділ 1: механічні та геометричні ефекти
	% =================================================
	\begin{frame}{%
			Аналіз впливу механічних та геометричних ефектів на процеси руйнування
			виготовленої з нелінійно-пружного матеріалу пластини скінченних розмірів
		}
		\small
		\begin{columns}[T]
			% -------- Ліва колонка: рисунок --------
			\begin{column}{0.5\textwidth}
				\centering
				\includegraphics[width=\textwidth]{Fig1.png}
			\end{column}
			
			% -------- Права колонка: висновки --------
			\begin{column}{0.5\textwidth}
				\begin{block}{Основний висновок}
					Для нелінійно-пружних матеріалів розтягування уздовж тріщини має
					незначний вплив на її кінцеве розкриття (механічний ефект),
					однак суттєво трансформує геометрію зони нелінійності
					перед фронтом тріщини.
					
					\medskip
					
					Це підкреслює необхідність урахування двовісного навантаження
					для точного визначення несучої здатності елементів конструкцій
					з тріщинами.
				\end{block}
			\end{column}
		\end{columns}
	\end{frame}
	
	% =================================================
	% Плакат 2.1: тільки рисунки (горизонтально)
	% =================================================
	\begin{frame}[t]{Часозалежне поширення просторової тріщини}
		
		\vspace{-1.0ex}
		\begin{figure}
			\centering
			\includegraphics[width=0.323\linewidth]{Stress0.png}\hfill
			\includegraphics[width=0.323\linewidth]{Stress1.png}\hfill
			\includegraphics[width=0.323\linewidth]{Stress2.png}
			
			\vspace{0.6ex}
			{\scriptsize Поле нормованих напружень поблизу фронту тріщини:
				$t=0$ с;\; $t=118{.}4$ с (завершення інкубації);\; $t=210{.}6$ с (перед ініціюванням динамічного руйнування).}
		\end{figure}
		
	\end{frame}
	
	
	% =================================================
	% Плакат 2.2: текст (методологія + висновки)
	% =================================================
	\begin{frame}[t]{Часозалежне поширення просторової тріщини}
		
\begin{block}{Коротко про підхід}
	Досліджено часову еволюцію тріщини у тривимірному в’язкопружному тілі
	в квазістатичній постановці з явним відстеженням фронту тріщини.
	Руйнування описано когезійним інтерфейсом
	із трапецеїдальним законом \textbf{зчеплення–відрив}.
\end{block}

\begin{block}{Урахування часової залежності}
	Часозалежність враховано шляхом \textbf{інкременталізації конститутивних співвідношень}:
	на кожному кроці оновлюються внутрішні змінні в’язкопружності,
	після чого розв’язується нелінійна задача рівноваги.
\end{block}

\begin{block}{Визначення фізичного часу}
	Для \textbf{заданої довжини тріщини} та \textbf{заданого рівня пошкодження}
	у вершині визначається відповідний фізичний час.
	Часовий приріст знаходиться з умови рівності зовнішнього навантаження
	та напруження, що забезпечує заданий рівень пошкодженості.
\end{block}

			
		
	\end{frame}
	
	% =================================================
	% Розділ 3 (компактно): крайова тріщина, напіваналітичні розв’язки
	% =================================================
	\begin{frame}[t]{Аналіз закономірностей руйнування елементів конструкцій внаслідок поширення крайової тріщини}
		
		\begin{columns}[T,onlytextwidth]
			% --------- Рисунок ----------
			\begin{column}{0.30\textwidth}
				\centering
				\includegraphics[width=\linewidth]{Fig2.png}
				
				{\footnotesize Нормоване розкриття $\bar{\Delta}(x,t)$ крайової тріщини.}
			\end{column}
			
			% --------- Текст (дуже стисло) ----------
			\begin{column}{0.68\textwidth}
				
				\begin{block}{Коротко про підхід}
					\small
					\begin{itemize}
						\item Розв’язки отримано \textbf{напіваналітично} на основі \textbf{сингулярних інтегральних рівнянь}.
						\item Невідома функція апроксимується та знаходиться \textbf{колокаційно} (зведення до СЛАР).
						\item Ядро інтегрального оператора 
						\(
						K(\xi,\tau)=\frac{1}{\tau-\xi}+\sum_{i=1}^{3}\frac{C_i(\xi)}{(\tau+\xi)^i},
						\)
					\end{itemize}
				\end{block}
				
				\begin{block}{Висновок}
					\small
					Часова еволюція профілю $\bar{\Delta}(x,t)$ дозволяє виявити характерні
					закономірності локалізації деформації біля фронту та параметри, що визначають
					умови поширення крайової тріщини.
				\end{block}
				
			\end{column}
		\end{columns}
		
	\end{frame}
	
% =================================================
% Публікації та літературна база
% =================================================
\begin{frame}{Публікації та літературна база}
	\small
	
	\begin{block}{Загальні показники}
		\begin{itemize}
			\item Загальна кількість публікацій — \textbf{23}
			\item У наукових журналах України — \textbf{8}
			\item У реферованих міжнародних виданнях — \textbf{15}
		\end{itemize}
	\end{block}
	
	\begin{block}{Розподіл реферованих публікацій за квартилями}
		\begin{itemize}
			\item \textbf{Q1} — 3 публікації
			\item \textbf{Q2–Q3} — 4 публікації
			\item \textbf{Q3–Q4} — 8 публікацій
		\end{itemize}
	\end{block}
	

\end{frame}

	
	% =================================================
	% Відгуки та рецензії (лише підписанти + афіліації)
	% =================================================
	\begin{frame}{Відгуки та рецензії}
		\small
		
		\textbf{Підписані відгуки/рецензії до звіту:}
		\vspace{0.6em}
		
		\begin{itemize}
			
			\item \textbf{Володимир Назаренко} — академік НАН України; академік-секретар відділення механіки і машинознавства НАНУ, в.о. директора Інституту механіки ім. С.П. Тимошенка НАНУ (представник замовника).
			
			\item \textbf{Володимир Козлов} — д-р фіз.-мат. наук; провідний науковий співробітник відділу термопружності,
			Інститут механіки ім. С.\,П. Тимошенка НАН України.
			
			\item \textbf{Артур Онищенко} — д-р техн. наук, професор; завідувач кафедри мостів, тунелів та гідротехнічних споруд,
			Національний транспортний університет.
			
			\item \textbf{Сергій Аксьонов} — канд. техн. наук, доцент; начальник відділу проєктування інфраструктурних об’єктів
			Управління відновлення та розвитку дорожньої інфраструктури 	Державного агентства відновлення та розвитку інфраструктури України.
			
		
		\end{itemize}
				
	\end{frame}
	
	
	\begin{frame}
		\frametitle{Офіційний відгук від Державного агентства відновлення та розвитку інфраструктури України}
		
		\begin{figure}
			\centering
			\begin{subfigure}[t]{0.35\textwidth}
				\centering
				\includegraphics[height=0.77\textheight]{Ukravtodor_beamer_left.png}
			\end{subfigure}
			\hspace{0.01\textwidth}
			\begin{subfigure}[t]{0.35\textwidth}
				\centering
				\includegraphics[height=0.77\textheight]{Ukravtodor_beamer_right.png}
			\end{subfigure}
		\end{figure}
		
	\end{frame}
	
	
	
	
	
\end{document}
