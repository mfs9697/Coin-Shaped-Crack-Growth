	\documentclass[12pt,a4paper]{article}
	
	\usepackage[margin=25mm]{geometry}
	\usepackage{amsmath,amssymb}
	\usepackage{graphicx}
	\usepackage{float}
	\usepackage{booktabs}
	\usepackage{subcaption}
	\usepackage{caption}
	\usepackage{siunitx}
	\usepackage{hyperref}
	
	\captionsetup{font=small,labelfont=bf}
	\hypersetup{
		colorlinks=true,
		linkcolor=blue,
		citecolor=blue,
		urlcolor=blue
	}
	
	\begin{document}
		
		%========================================================
		\section{Time-dependent propagation of a coin-shaped crack}
		\label{sec:coin_crack}
		
		In this section, the propagation of a coin-shaped (penny-shaped) crack in a
		three-dimensional viscoelastic solid is investigated using a cohesive-zone-based
		finite-element formulation. Crack growth is driven by stress relaxation under a
		prescribed external loading and is described as a sequence of quasi-static
		equilibrium states with an explicitly tracked crack front.
		
		The analysis exploits symmetry and considers a quarter of the crack embedded in
		a cubic block. A locally refined finite-element mesh is introduced in an annular
		region surrounding the crack front, while coarser elements are used away from
		this zone. A cohesive interface with a finite process zone is embedded along the
		crack surface and governed by a smoothed trapezoidal traction--separation law.
		
		% -------------------------------------------------
		\subsection{Crack-growth criterion and control parameter}
		\label{subsec:criterion_kD}
		
		Crack propagation is controlled by a local opening condition evaluated at the
		current crack-front node. At each propagation step, the following criterion is
		enforced:
		\begin{equation}
			\frac{2u_z}{\Delta_{\max}} = k_D ,
			\label{eq:opening_criterion}
		\end{equation}
		where $u_z$ is the normal opening displacement at the active crack-front node,
		$\Delta_{\max}$ is the critical cohesive opening displacement, and
		$k_D \in (0,1]$ is a prescribed damage parameter.
		
		The parameter $k_D$ is increased incrementally. Once $k_D = 1$ is reached, the
		crack front advances to the next node along the predefined crack path. This
		strategy provides a controlled and stable simulation of crack-front evolution
		without introducing ad hoc remeshing or heuristic growth rules. In the present
		implementation, the crack-front advance is realized by updating the active
		cohesive index (front node) and continuing the damage-controlled evolution for
		the next segment of the predefined front.
		
		% -------------------------------------------------
		\subsection{Determination of the physical time increment}
		\label{subsec:dt_fzero}
		
		A central aspect of the algorithm is the determination of the physical time
		increment corresponding to a prescribed external stress level. Instead of
		introducing an artificial time step, the increment $\Delta t$ is computed
		implicitly from the nonlinear condition
		\begin{equation}
			\sigma_z(\Delta t) = \sigma^{\mathrm{ext}},
			\label{eq:time_condition}
		\end{equation}
		where $\sigma^{\mathrm{ext}}$ is the applied external stress and $\sigma_z$ is the
		stress response at the crack front accounting for viscoelastic relaxation.
		
		Equation~\eqref{eq:time_condition} is solved using the MATLAB root-finding routine
		\texttt{fzero}. For a given trial value of $\Delta t$, the viscoelastic internal
		variables are updated analytically using exponential relaxation kernels, the
		tangent stiffness matrix is reconstructed, and the global equilibrium problem is
		solved. The resulting stress level is then compared with
		$\sigma^{\mathrm{ext}}$, and the root of
		$\sigma_z(\Delta t) - \sigma^{\mathrm{ext}} = 0$ is determined.
		
		To ensure robustness, the root is bracketed automatically by scanning $\Delta t$
		over a small set of logarithmically spaced trial values. The first sign change of
		$g(\Delta t)=\sigma_z(\Delta t)-\sigma^{\mathrm{ext}}$ defines a valid bracket
		interval for \texttt{fzero}. In the reported computations, this bracketing
		strategy resulted in stable convergence, with residuals reduced to machine
		precision and without sensitivity to the initial guess. Importantly, the computed
		$\Delta t$ provides a physically meaningful time scale governed directly by the
		material relaxation properties.
		
		% -------------------------------------------------
		\subsection{Displacement and stress profiles at the crack surface}
		\label{subsec:profiles}
		
		The evolution of the cohesive zone is conveniently illustrated through profiles
		along the crack surface. At each saved step, we plot the normalized crack opening
		$2u_z/\Delta_{\max}$ and the normalized normal stress $\sigma_z/\sigma_{\max}$ as
		functions of the radial coordinate $x$ measured along the crack surface. The
		current crack radius $R_c$ is indicated by the vertical dashed line.
		
		A practical observation in three-dimensional cohesive simulations is that for
		shallow meshes the raw stress component $\sigma_z$ may exhibit unphysical sign
		changes near the front due to discretization effects. To avoid misleading trends,
		we therefore present the \emph{normalized} stress $\sigma_z/\sigma_{\max}$, which
		remains within $[0,1]$ and enables consistent comparison between time steps.
		
		Figure~\ref{fig:profiles_grid} summarizes ten steps for which the cohesive zone
		remains fully inside the refinement ring. The plots demonstrate:
		(i) a strong localization of the opening field near the crack front,
		(ii) a pronounced, smooth stress peak within the cohesive/process zone, and
		(iii) a systematic outward shift of the profiles as the crack advances.
		
		\begin{figure}[H]
			\centering
			\captionsetup{justification=centering}
			\begin{subfigure}[t]{0.4\textwidth}
				\centering
				\includegraphics[width=\textwidth]{Fig2_k000_t0.png}
			\end{subfigure}\hfill
			\begin{subfigure}[t]{0.4\textwidth}
				\centering
				\includegraphics[width=\textwidth]{Fig2_k001_t49.3.png}
				\caption{$k=1$, $t=49.3$ s}
			\end{subfigure}
			
			\begin{subfigure}[t]{0.4\textwidth}
				\centering
				\includegraphics[width=\textwidth]{Fig2_k002_t88.05.png}
			\end{subfigure}\hfill
			\begin{subfigure}[t]{0.4\textwidth}
				\centering
				\includegraphics[width=\textwidth]{Fig2_k003_t118.4.png}
			\end{subfigure}
			
			\begin{subfigure}[t]{0.4\textwidth}
				\centering
				\includegraphics[width=\textwidth]{Fig2_k004_t143.5.png}
			\end{subfigure}\hfill
			\begin{subfigure}[t]{0.4\textwidth}
				\centering
				\includegraphics[width=\textwidth]{Fig2_k005_t164.3.png}
			\end{subfigure}
			
			\begin{subfigure}[t]{0.4\textwidth}
				\centering
				\includegraphics[width=\textwidth]{Fig2_k006_t181.7.png}
			\end{subfigure}\hfill
			\begin{subfigure}[t]{0.4\textwidth}
				\centering
				\includegraphics[width=\textwidth]{Fig2_k007_t194.8.png}
			\end{subfigure}
			
			\begin{subfigure}[t]{0.4\textwidth}
				\centering
				\includegraphics[width=\textwidth]{Fig2_k008_t204.4.png}
			\end{subfigure}\hfill
			\begin{subfigure}[t]{0.4\textwidth}
				\centering
				\includegraphics[width=\textwidth]{Fig2_k009_t210.6.png}
			\end{subfigure}
			
			\caption{Normalized cohesive profiles along the crack surface at ten successive
				steps while the cohesive/process zone remains inside the refinement ring.
				Each panel shows $2u_z/\Delta_{\max}$ and $\sigma_z/\sigma_{\max}$ versus $x$,
				with the current crack radius $R_c$ marked by the vertical dashed line.}
			\label{fig:profiles_grid}
		\end{figure}
		
		% -------------------------------------------------
		\subsection{Time--radius relationship}
		\label{subsec:time_radius}
		
		The evolution of the crack radius $a(t)$ obtained from the proposed algorithm is
		shown in Fig.~\ref{fig:time_radius}. An initial incubation stage with a
		stationary crack is followed by a phase of accelerated crack growth once the
		critical opening condition~\eqref{eq:opening_criterion} is satisfied.
		This behavior emerges naturally from the coupled solution of the viscoelastic
		constitutive equations, the cohesive-zone model, and the implicit determination
		of the time increment. No additional assumptions regarding the crack-growth rate
		are introduced.
		
		\begin{figure}[H]
			\centering
			\includegraphics[width=0.50\textwidth]{Time-Radius.png}
			\caption{Time--radius curve $a(t)$ for the coin-shaped crack. The incubation
				stage (no growth) is followed by stable quasi-static propagation driven by
				viscoelastic relaxation under sustained loading.}
			\label{fig:time_radius}
		\end{figure}
		
		For transparency, Table~\ref{tab:time_radius_points} summarizes the crack radius
		recorded at the ten saved steps. The crack remains at $a_0=3.0$~cm during the
		early part of the simulation and subsequently increases monotonically as the
		front advances in the refined ring.
		
		\begin{table}[H]
			\centering
			\caption{Recorded crack radius $a(t)$ at the saved steps corresponding to
				Fig.~\ref{fig:profiles_grid}.}
			\label{tab:time_radius_points}
			\small
			\begin{tabular}{@{}rS[table-format=3.1]S[table-format=1.4]@{}}
				\toprule
				{$k$} & {$t$ [s]} & {$a(t)$ [cm]} \\
				\midrule
				0 & 0.0   & 3.0000 \\
				1 & 49.3  & 3.0000 \\
				2 & 88.1  & 3.0000 \\
				3 & 118.4 & 3.0000 \\
				4 & 143.5 & 3.0938 \\
				5 & 164.3 & 3.1875 \\
				6 & 181.7 & 3.2813 \\
				7 & 194.8 & 3.3750 \\
				8 & 204.4 & 3.4688 \\
				9 & 210.6 & 3.5625 \\
				\bottomrule
			\end{tabular}
		\end{table}
		
		% -------------------------------------------------
		\subsection{Stress field near the crack front: initial, incubation, and growth stages}
		\label{subsec:stress_fields}
		
		To visualize the spatial evolution of the fracture process zone, we plot the
		normalized stress field $\sigma_z/\sigma_{\max}$ at three characteristic time
		instants: (i) $t=0$~s (initial), (ii) $t=118.4$~s (end of incubation), and
		(iii) $t=210.6$~s (advanced growth). The results are shown in
		Fig.~\ref{fig:stress_fields_3}.
		
		At $t=0$~s, the stress concentration is localized near the initial crack front,
		and the cohesive annulus is confined to the refined ring. During the incubation
		stage, the crack radius remains unchanged; nevertheless, viscoelastic relaxation
		redistributes stresses in the vicinity of the front. At $t=118.4$~s, the stress
		field indicates that the system approaches the critical configuration at which
		the cohesive condition~\eqref{eq:opening_criterion} becomes active. Finally, at
		$t=210.6$~s, the zone of elevated stresses has shifted outward consistently with
		the increased crack radius, demonstrating stable quasi-static propagation with a
		smooth stress gradient across the cohesive/process zone.
		
		\begin{figure}[H]
			\centering
			\begin{subfigure}[t]{0.32\textwidth}
				\centering
				\includegraphics[width=\textwidth]{Stress0.png}
				\caption{$t=0$ s}
			\end{subfigure}\hfill
			\begin{subfigure}[t]{0.32\textwidth}
				\centering
				\includegraphics[width=\textwidth]{Stress1.png}
				\caption{$t=118.4$ s (incubation)}
			\end{subfigure}\hfill
			\begin{subfigure}[t]{0.32\textwidth}
				\centering
				\includegraphics[width=\textwidth]{Stress2.png}
				\caption{$t=210.6$ s (growth)}
			\end{subfigure}
			\caption{Normalized stress field $\sigma_z/\sigma_{\max}$ near the crack front
				at three characteristic time instants: initial state, end of incubation, and
				advanced propagation. The annular zone of elevated stress tracks the crack edge,
				confirming consistent front evolution and adequate resolution of the cohesive
				process zone by the refinement ring.}
			\label{fig:stress_fields_3}
		\end{figure}
		
		\noindent
		Overall, the combined analysis of profiles (Fig.~\ref{fig:profiles_grid}),
		the time--radius curve (Fig.~\ref{fig:time_radius}), and the stress-field
		snapshots (Fig.~\ref{fig:stress_fields_3}) confirms that crack advance in the
		present problem is governed by a time-dependent mechanism: a delayed activation
		(incubation) followed by stable quasi-static propagation driven by viscoelastic
		relaxation under sustained loading.
		
	\end{document}
	
	
