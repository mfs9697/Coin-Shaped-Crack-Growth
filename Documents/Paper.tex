% Generated by GrindEQ Word-to-LaTeX 
\documentclass{article} % use \documentstyle for old LaTeX compilers

\usepackage[utf8]{inputenc} % 'cp1252'-Western, 'cp1251'-Cyrillic, etc.
\usepackage[english]{babel} % 'french', 'german', 'spanish', 'danish', etc.
\usepackage{amsmath}
\usepackage{amssymb}
\usepackage{txfonts}
\usepackage{mathdots}
\usepackage[classicReIm]{kpfonts}
\usepackage{graphicx}

%%% remove comment delimiter ('%') and select graphics package
%%% for DVI output:
\usepackage[dvips]{graphicx}
%%% or for PDF output:
%\usepackage[pdftex]{graphicx}
%%% or for old LaTeX compilers:
%\usepackage[dvips]{graphics}


\begin{document}

%\selectlanguage{english} % remove comment delimiter ('%') and select language if required


\noindent 
\title{}\maketitle 

\noindent 
\title{}\maketitle 

\noindent 
\title{}\maketitle 

\noindent 
\title{
\[58\] }\maketitle 


\noindent 

\textbf{3ВСТУП}



В'язкопружні композитні матеріали, такі як полімери та композити на їх основі, відіграють суттєву роль у сучасних галузях промисловості, таких як авіакосмічна, автомобільна та цивільне будівництво. Низка важливих механічних властивостей серед яких висока міцність, довговічність, здатність розсіювати енергію та витримувати тривалу деформацію під впливом різних типів навантажень, робить їх незамінними для практичного застосування в якості конструкційних матеріалів. Залежна від часу в'язкопружна поведінка даних матеріалів, дозволяє їм поглинати енергію удару, зменшувати вібрацію та адаптуватися до змін навколишнього середовища, таких як температура та вологість. Їх в'язкопружна природа є особливо вигідною в тих випадках, коли довговічність і надійність є критичними, наприклад, у компонентах літаків, безпілотних летальних апаратах та лопатях вітрових турбін. Точне моделювання їх довготривалої поведінки, що включаючи повзучість і втому, дозволяє прогнозувати термін служби та оптимізувати конструкцію деталей машин, що застосовуються для конкретних потреб.

Необхідність врахування в'язкопружних властивостей при проєктуванні деталей конструкцій обумовлена тим фактом, що навіть безпечні навантаження, обчислені без врахування спадкових властивостей матеріалу, при довготривалому впливі можуть призвести до поступового перерозподілу напруження, виникненню додаткових деформацій та динамічного руйнуванню через унікальну залежну від часу поведінку в'язкопружних матеріалів. Докритичне навантаження може призвести до раптового та непередбачуваного руйнування після певної затримки часу, протягом якої не спостерігається видимих ознак неминучої дезінтеграції конструкції. Ця затримка, яку зазвичай називають відтермінованим руйнуванням, виникає внаслідок взаємопов'язаних ефектів повзучості та релаксації напружень --- двох фундаментальних спадкових властивостей в'язкопружних матеріалів, які невід'ємно пов'язані в лінійній теорії в'язкопружності. Ці властивості визначають зміну напружень з часом, сприяючи як перерозподілу навантажень, так і поступовій деформації матеріалу. Наявність концентраторів напружень, таких як отвори або тріщини, ще більше прискорює локалізоване накопичення пошкоджень шляхом посилення напруження в певних ділянках конструкції. Згодом ці накопичені пошкодження можуть послабити матеріал до критичного порогу, викликаючи катастрофічну руйнацію навіть за, здавалося б, безпечних умов навантаження. Розпізнавання та розуміння цих явищ має важливе значення для точного прогнозування довгострокової поведінки та надійності в'язкопружних конструкційних матеріалів, що працюють при докритичних з точки зору пружних властивостей навантаженнях.

Залежно від співвідношення прикладеного навантаження до критичного значення, при якому тріщина виникає миттєво, час затримки між прикладанням навантаження та руйнуванням може становити від секунд, годин, днів або навіть років, що підкреслює складну взаємодію між властивостями матеріалу, геометричними факторами та умовами навантаження. Тому дослідження проблем в'язкопружного руйнування має важливе значення для прогнозування довгострокової надійності та безпеки конструкцій.

Дослідження довготривалого руйнування почалися в 1960-х роках, а сучасні концепції моделювання були розроблені на початку 1980-х років. Ці концепції в першу чергу спираються на модель когезійної зони, яка усуває нескінченні напруження у вершині тріщини, забезпечуючи можливість застосування теорії лінійної пружності. Порівняння основних теорій поширення тріщин можна знайти в [1], де також можна прослідкувати історію розвитку цього напряму досліджень.

При використанні моделі когезійної зони для моделювання зародження тріщини часто виникають труднощі з досягненням збіжності в точці, де зароджується тріщина. Ці проблеми пов'язані із поновлювальною нестабільністю, яка виникає відразу після того, як напруження досягає величини когезійної міцності. Для вирішення цих проблем збіжності були запропоновані різні підходи. Визначення розкриття в вершині тріщини при використанні інтенсивності зовнішнього навантаження як змінної, що стабілізує розв'язок, гарантуючи, що розкриття в вершині тріщини залишається в фізично реалістичних межах, тоді як інтенсивність навантаження регулюється для досягнення рівноваги. Іншими словами, неможливо знайти переміщення розкриття вершини тріщини для заданого рівня навантаження, але зворотна процедура не має проблем збіжності. Введення малої в'язкості в основні рівняння також допомагає врегулювати нестабільність шляхом пом'якшення різких змін напружень, що робить його особливо корисним для матеріалів із залежними від часу властивостями. Загальні схеми, такі як метод Рікса [2] та його модифікація [3], пропонують ітераційні методи, які відстежують нестабільну гілку розв'язку, забезпечуючи гнучкість у вирішенні різноманітних сценаріїв навантаження та матеріалів. 

Інший випадок відтермінованого руйнування пов'язаний із зародженням тріщини в точці досягнення напруженням критичного значення та поступовим переходом до динамічного режиму або припиненням еволюції ушкодження. Цей сценарій є особливо складним для прогнозування структурної цілісності, оскільки він включає кілька складних механізмів. Крім того, взаємодія між силами когезії та перерозподілом напружень у навколишньому матеріалі додає ще один рівень складності. Розуміння цього вимагає детального вивчення кінетики поширення тріщини, включаючи такі фактори, як залежні від часу зміни властивостей матеріалу та еволюція поля напружень поблизу вершини тріщини.

Підхід, запропонований у [5], започаткував дослідження квазістатичного розповсюдження тріщин у в'язкопружному ортотропному середовищі. Цей підхід передбачає наявність невеликої зони передруйнування або початкового руйнування поблизу фронту тріщини, де відбувається часткова руйнація матеріалу. Фундаментальна робота базується на інтегральній формі конститутивних співвідношень Больцмана--Вольтерра та моделі Баренблата--Дагдейла [6,7], яка передувала розвитку моделі когезійної зони [8]. 

Відповідно до закону зчеплення--відриву (ЗЗВ), когезійні сили пов'язані із розкриттям уздовж когезійної зони. Закон росту тріщини отримується як розв'язок інтегрального рівняння, яке включає функції повзучості, що описують розкриття тріщини яке змінюється з часом. Ці функції можна отримати через пружно--в'язкопружну аналогію, використовуючи відповідний пружний розв'язок задачі. Такий підхід в основному використовується в аналітичних розв'язках задач механіки тріщин.

У числових дослідженнях в'язкопружних матеріалів, широко застосовується метод інкременталізації конститутивних співвідношень у поєднанні з пружно-в'язкопружною аналогією. Огляд підходів скінченноелементного моделювання для ізотропних і трансверсально-ізотропних в'язкопружних матеріалів надано в [9]. Підхід внутрішньої змінної, який вимагає маніпулювання величинами, визначеними в точках інтегрування, був запропонований для реалізації в рамках методу скінчених елементів, розширюючи його можливості на визначення напружено-деформованого стану елементів конструкції з лінійно в'язкопружних матеріалів. Підхід внутрішньої змінної вперше був застосований до ортотропних в'язкопружних матеріалів у [10]. Компоненти визначального рівняння в інтегральній формі переведено в інкрементну форму, що і дозволило отримати розв'язок лінійних задач в'язкопружності методом скінченних елементів. Цей метод використовувався для дослідження концентрації напружень біля отворів у в'язкопружних пластинах в умовах плоского напруженого стану [11] і відтермінованого руйнування плоских [12] і просторових [13] тіл з наявною до прикладання навантаження тріщиною, яке передує динамічному поширенню тріщини [14,15]. 

\eject \textbf{1  РОЗРОБКА МЕТОДУ МОДЕЛЮВАННЯ НАПРУЖЕНО-ДЕФОРМОВАНОГО СТАНУ В'ЯЗКОПРУЖНИХ ЕЛЕМЕНТІВ КОНСТРУКЦІЇ З КОНЦЕНТРАТОРАМИ НАПРУЖЕНЬ}

\textbf{}

\textbf{1.1 Інтеграція методів дослідження напружено-деформованого стану із моделями розвитку тріщин у в'язкопружних елементах конструкції з концентраторами}

\textbf{}

Явище зростання тріщин у в'язкопружних матеріалах було вперше описано в рамках моделі Баренблатта--Дагдейла в [13] на основі рівняння енергетичного балансу:

\noindent 
\title{
\begin{equation} \label{GrindEQ__1_1_} 
\int _{\lambda }^{\beta } \sigma (x_{1} ,t)\frac{\partial \Delta (x_{1} ,t)}{\partial t} {\rm d}x_{1} =\phi \frac{{\rm d}\lambda (t)}{{\rm d}t} , 
\end{equation} }\maketitle 

\title{}\maketitle 

\noindent де $t$ -- час, $\lambda $ -- довжина тріщини, $\sigma $-- сила зчеплення (когезії), $\Delta $ -- в'язкопружне розкриття вздовж когезійної зони і $\phi $ -- енергія руйнування. Залежно від моделі тріщини когезія може залежати як від координати, так і від розкриття. Положення вершини когезійної зони в кожній момент часу має визначатися умовою скінченності напружень в вершині когезійної зони. Таким чином, задача зводиться до аналізу локального енергетичного балансу в зоні трансформації.

Перехід до рухомої системи координат із розкриттям, незалежним від часу, дає:

\noindent 
\title{
\begin{equation} \label{GrindEQ__1_2_} 
\int _{0}^{\Delta _{\max } } \sigma (x_{1} ,t){\rm d}\Delta =\phi , 
\end{equation} }\maketitle 

\title{}\maketitle 

\noindent де $\Delta _{\max } $ -- розкриття в вершині фізичної тріщини, що відповідає стану граничної рівноваги. Критерій руйнування \eqref{GrindEQ__1_2_}, який часто називають критерієм критичної когезійної роботою, часто використовується при моделюванні поширення тріщин.

 Конститутивні співвідношення Больцмана--Вольтерра для лінійного в'язкопружного матеріалу, що не старіє, в матричній формі мають вигляд


\begin{equation} \label{GrindEQ__1_3_} 
\sigma _{i} =\int _{-\infty }^{t} C_{ik} (t-\tau )\frac{\partial \varepsilon _{k} }{\partial \tau } {\rm d}\tau , 
\end{equation} 



\noindent де $\sigma _{i} =\sigma _{i} (\boldsymbol{\mathrm{x}},t)$ та $\varepsilon _{k} =\varepsilon _{k} (\boldsymbol{\mathrm{x}},t)$ -- компоненти векторів


\begin{equation} \label{GrindEQ__1_4_} 
\boldsymbol{\mathrm{\sigma }}=\{ \sigma _{11} ,\sigma _{22} ,\sigma _{12} \} ^{{\rm T}} ,\quad \boldsymbol{\mathrm{\varepsilon }}=\{ \varepsilon _{11} ,\varepsilon _{22} ,\varepsilon _{12} \} ^{{\rm T}} , 
\end{equation} 


\[\boldsymbol{\mathrm{x}}=(x_{1} ,x_{2} ). \] 

\title{}\maketitle 

Ненульові компоненти симетричної матриці $\boldsymbol{\mathrm{C}}$ можна отримати, використовуючи принцип пружно--в'язкопружної аналогії, з виразів для відповідних пружних величин 


\begin{equation} \label{GrindEQ__1_5_} 
C_{11} =\frac{E_{11} }{1-\nu _{12} \nu _{21} } ,\quad C_{12} =\frac{\nu _{12} E_{22} }{1-\nu _{12} \nu _{21} } ,\quad C_{22} =\frac{E_{22} }{1-\nu _{12} \nu _{21} } ,\quad C_{33} =G_{12}  
\end{equation} 

\[ (\nu _{12} E_{22} =\nu _{21} E_{11} ). \] 

\title{}\maketitle 

Якщо функції релаксації матеріалу визначено в формі ряду Проні--Діріхле (далі будемо вказувати індекси, за якими йде сумування; за іншими індексами, що повторюються, сумування немає)


\begin{equation} \label{GrindEQ__1_6_} 
C_{ik} (t)=E_{ik}^{\infty } +\sum _{m} E_{ik}^{(m)} \exp \left\{-t/\rho _{ik}^{(m)} \right\}.  
\end{equation} 

\title{}\maketitle 

За $m=1$ і $\rho _{ik}^{\eqref{GrindEQ__1_}} =\rho $, аналітичні вирази для компонент конститутивної матриці методу скінченних елементів отримаємо у вигляді: 


\begin{equation} \label{GrindEQ__1_7_} 
\begin{array}{rrr} {D_{11} (t)=} & {\frac{E_{1}^{\infty } E_{2}^{\infty } }{E_{\alpha }^{\infty } } +\frac{E_{1}^{1} E_{2}^{1} }{E_{\alpha }^{1} } \exp \left\{-\frac{t}{\rho } \right\}-} & {\frac{\alpha (E_{2}^{0} E_{1}^{\infty } -E_{1}^{1} E_{2}^{\infty } )^{2} }{E_{\alpha }^{\infty } E_{\alpha }^{1} E_{\alpha }^{0} } \exp \left\{-\frac{E_{\alpha }^{\infty } }{E_{\alpha }^{0} } \frac{t}{\rho } \right\},} \\ {D_{22} (t)=} & {\frac{(E_{2}^{\infty } )^{2} }{E_{\alpha }^{\infty } } +\frac{(E_{2}^{1} )^{2} }{E_{\alpha }^{1} } \exp \left\{-\frac{t}{\rho } \right\}-} & {\frac{\alpha ^{2} (E_{2}^{0} E_{1}^{\infty } -E_{1}^{1} E_{2}^{\infty } )^{2} }{E_{\alpha }^{\infty } E_{\alpha }^{1} E_{\alpha }^{0} } \exp \left\{-\frac{E_{\alpha }^{\infty } }{E_{\alpha }^{0} } \frac{t}{\rho } \right\},} \end{array} 
\end{equation} 
\[D_{12} (t)=\nu _{21} D_{11} (t),\quad D_{33} (t)=G_{12} (t),\] 

\title{}\maketitle 

\noindent де


\begin{equation} \label{GrindEQ__1_8_} 
\begin{array}{c} {E_{\alpha }^{0} =E_{2}^{0} -\alpha E_{1}^{0} ,\quad E_{\alpha }^{\infty } =E_{2}^{\infty } -\alpha E_{1}^{\infty } ,\quad E_{\alpha }^{1} =E_{2}^{1} -\alpha E_{1}^{1} ,\quad \alpha =\nu _{21}^{2} ,} \\ {E_{1}^{0} =E_{1}^{\infty } +E_{1}^{1} ,\quad E_{2}^{0} =E_{2}^{\infty } +E_{2}^{1} .} \end{array} 
\end{equation} 

\title{}\maketitle 

Для імплементації в'язкопружної моделі в розрахункову схему метода скінченних елементів, напруження в поточний момент часу записуються у вигляді


\begin{equation} \label{GrindEQ__1_9_} 
\boldsymbol{\mathrm{\sigma }}^{(n)} =\boldsymbol{\mathrm{\sigma }}^{(n-1)} -\tilde{\boldsymbol{\mathrm{\sigma }}}^{(n)} +\hat{\boldsymbol{\mathrm{\sigma }}}^{(n)} , 
\end{equation} 

\title{}\maketitle 

\noindent де $\boldsymbol{\mathrm{\sigma }}^{(n)} =\boldsymbol{\mathrm{\sigma }}(\boldsymbol{\mathrm{x}},t_{n} )$, $\tilde{\boldsymbol{\mathrm{\sigma }}}^{(n)} =\tilde{\boldsymbol{\mathrm{\sigma }}}(\boldsymbol{\mathrm{x}},t_{n} )$ ($\tilde{\boldsymbol{\mathrm{\sigma }}}$ -- внутрішня змінна задачі), $\hat{\boldsymbol{\mathrm{\sigma }}}^{(n)} =\hat{\boldsymbol{\mathrm{\sigma }}}(\boldsymbol{\mathrm{x}},t_{n} )$,


\[\tilde{\sigma }_{i}^{(n)} =-\int _{-\infty }^{t_{n-1} } \left[C_{ik} (t_{n} -\tau )-C_{ik} (t_{n-1} -\tau )\right]\frac{\partial \varepsilon _{k} }{\partial \tau } {\rm d}\tau ,\] 
\begin{equation} \label{GrindEQ__1_10_} 
\hat{\sigma }_{i}^{(n)} =\int _{t_{n-1} }^{t_{n} } C_{ik} (t_{n} -\tau )\frac{\partial \varepsilon _{k} }{\partial \tau } {\rm d}\tau . 
\end{equation} 



Будемо визначати вектор інкременту переміщень $\Delta \boldsymbol{\mathrm{u}}^{(n)} =\boldsymbol{\mathrm{N}}^{{\rm T}} \Delta \boldsymbol{\mathrm{q}}^{(n)} $ та деформацій $\Delta \boldsymbol{\mathrm{\varepsilon }}^{(n)} =\boldsymbol{\mathrm{B}}^{{\rm T}} \Delta \boldsymbol{\mathrm{q}}^{(n)} $ в елементі з дискретизованої системи алгебраїчних рівнянь, яка розв'язується для невідомих приростів вузлових переміщень, $\Delta \boldsymbol{\mathrm{q}}^{(n)} $:


\begin{equation} \label{GrindEQ__1_11_} 
\boldsymbol{\mathrm{K}}\Delta \boldsymbol{\mathrm{q}}^{(n)} =\boldsymbol{\mathrm{F}}^{t} -\boldsymbol{\mathrm{F}}^{(n-1)} +\tilde{\boldsymbol{\mathrm{F}}}^{(n)} , 
\end{equation} 



\noindent де


\[\boldsymbol{\mathrm{K}}^{(n)} =\int _{\Omega ^{e} } \boldsymbol{\mathrm{B}}^{{\rm T}} \tilde{\boldsymbol{\mathrm{E}}}^{(n)} \boldsymbol{\mathrm{B}}{\rm d}V,\] 
\begin{equation} \label{GrindEQ__1_12_} 
\boldsymbol{\mathrm{F}}^{t} =\int _{\partial \Omega ^{e} } \boldsymbol{\mathrm{N}}^{{\rm T}} \boldsymbol{\mathrm{T}}{\rm d}S,\quad \boldsymbol{\mathrm{F}}^{(n)} =\int _{\Omega ^{e} } \boldsymbol{\mathrm{B}}^{{\rm T}} \boldsymbol{\mathrm{\sigma }}^{(n)} {\rm d}V,\quad \tilde{\boldsymbol{\mathrm{F}}}^{(n)} =\int _{\Omega ^{e} } \boldsymbol{\mathrm{B}}^{{\rm T}} \tilde{\boldsymbol{\mathrm{\sigma }}}^{(n)} {\rm d}V, 
\end{equation} 



\noindent $\boldsymbol{\mathrm{B}}$ -- стандартна матриця деформації--переміщення, яка пов'язує деформації з вектором вузлового переміщення, $\boldsymbol{\mathrm{T}}$ -- вектор поверхневих сил, $\boldsymbol{\mathrm{N}}$ -- вектор функцій форм, $\boldsymbol{\mathrm{K}}^{(n)} $ -- інкрементна матриця жорсткості, $\Omega ^{e} $ -- об'єм елемента, $\partial \Omega ^{e} $ -- його границя.

Якщо функції релаксації матеріалу визначено в формі ряду Проні--Діріхле \eqref{GrindEQ__6_}, то внутрішню змінну можна визначати рекурсивно:


\begin{equation} \label{GrindEQ__1_13_} 
\tilde{\sigma }_{i}^{(n)} =\sum _{m} \sum _{k} \left(1-\zeta _{ik}^{(m,n)} \right)S_{ik}^{(m,n)} ,\quad n=1,\ldots  
\end{equation} 


\[S_{ik}^{(m,0)} =E_{ik}^{(m)} \varepsilon _{k}^{(0)} ,\] 
\begin{equation} \label{GrindEQ__1_14_} 
S_{ik}^{(m,n)} =\zeta _{ik}^{(m,n)} S_{ik}^{(m,n)} +\eta _{ik}^{(m,n)} \left(1-\zeta _{ik}^{(m,n)} \right)\Delta \varepsilon _{k}^{(n)} ,\quad n=1,2,\ldots  
\end{equation} 



\noindent  елементи матриці $\tilde{\boldsymbol{\mathrm{E}}}^{(n)} $,


\begin{equation} \label{GrindEQ__1_15_} 
\tilde{E}_{ik}^{(n)} =E_{ik}^{\infty } +\sum _{m} \eta _{ik}^{(m,n)} \left(1-\zeta _{ik}^{(m,n)} \right), 
\end{equation} 
\begin{equation} \label{GrindEQ__1_16_} 
\zeta _{ik}^{(m,n)} =\exp \left\{-\Delta t_{n} /\rho _{ik}^{(m)} \right\},\quad \eta _{ik}^{(m,n)} =\frac{E_{ik}^{(m)} \rho _{ik}^{(m)} }{\Delta t_{n} } . 
\end{equation} 



Величини $S_{ik}^{(m,n)} $ рекурентно визначаються в точках інтегрування на кожному часовому інтервалі. 

 Пряма задача дослідження розвитку тріщини полягає у знаходженні конфігурації розкриття для заданого рівня зовнішнього навантаження. Такий підхід можна реалізувати лише для незначних рівнів розкриття у вершині тріщини порівняно з критичним значенням. В іншому випадку треба розглядати обернену задачу -- для заданого розкриття в вершині знаходити відповідний йому рівень зовнішнього навантаження. 

\textbf{Пряма задача:} Для заданого рівня зовнішнього навантаження $p$ знайти поле переміщень $\boldsymbol{\mathrm{u}}$, що задовольняє ЗЗВ $\boldsymbol{\mathrm{\sigma }}^{{\rm (coh)}} =T(2\boldsymbol{\mathrm{u}}^{{\rm (coh)}} )$, де $\boldsymbol{\mathrm{\sigma }}^{{\rm (coh)}} $ та $\boldsymbol{\mathrm{u}}^{{\rm (coh)}} $ є силами когезії та вертикальними переміщеннями вздовж лінії когезії, $T$ є заданим ЗЗВ.

\textbf{Обернена задача: }Для заданого розкриття в вершині тріщини $\Delta $ знайти поле переміщень $\boldsymbol{\mathrm{u}}$, що задовольняє ЗЗВ: $\boldsymbol{\mathrm{\sigma }}^{{\rm (coh)}} =T(2\boldsymbol{\mathrm{u}}^{{\rm (coh)}} )$, та відповідний рівень зовнішнього навантаження $p$.

\noindent 
\title{}\maketitle 

\textbf{1.2 Алгоритм побудови залежності довжини тріщини від часу}

\noindent 
\title{}\maketitle 

Розглянемо два випадки -- наявність при прикладанні до тіла навантаження тріщини або отвору. Перший випадок вимагає введення зони зчеплення для обмеження напруження в околі вершини. В другому випадку зона зчеплення вводиться коли напруження в концентраторі більше за міцність зчеплення -- починаючи з концентратору в напрямку поширення. У випадку круглого отвору в ізотропній пластині концентрація не збільшується з часом -- відтерміноване руйнування зініціюється в момент прикладання навантаження. У випадку ортотропії властивостей матеріалу концентрація напружень може сягнути критичного значення протягом деякого часу після прикладання докритичного для пружного випадку навантаження. 

Якщо внаслідок розв'язання пружної задачі відносне розкриття в вершині тріщини або в концентраторі ($\bar{\Delta }_{0} =\Delta (\lambda _{0} )/\Delta _{\max } $) не перевищує одиниці починається дослідження відтермінованого руйнування. Інкубаційний період розвитку тріщини характеризується збільшенням розкриття в її вершині без зміни початкової довжини $\lambda _{0} $ (у випадку концентратора $\lambda _{0} =0$). Таким чином, при прикладанні докритичного (для відповідної пружної задачі) рівня навантаження деякий час не спостерігається зростання фізичної тріщини. За обмеженої повзучості може існувати такий рівень навантаження, за якого розвиток тріщини не відбуватиметься взагалі. Розділимо інтервал $(\bar{\Delta }_{0} ,1)$ на частини і будемо послідовно визначати прирости часу, що відповідають кожному інкременту розкриття. Час, який відповідає останньому приросту, визначатиме тривалість інкубаційного періоду зростання тріщини (або її зародження у випадку концентратора). Далі починається дослідження зростання тріщини. Для кожного інкременту довжини тріщини визначатиме відповідний приріст часу. 

В основі алгоритму лежить використання на кожному $n$-му кроці функції $\Sigma (\Delta t_{n} ,\Delta _{n} ,\lambda _{n} )$, яка визначає напруження, що необхідно прикласти до тіла щоб за час $\Delta t_{n} $ після попереднього визначеного часу розкриття в вершині тріщини було рівно $\Delta _{n} $, а довжина -- $\lambda _{n} $. Під час інкубації $\Delta _{n} <\Delta _{\max } $ і $\lambda _{n} =\lambda _{0} $, під час поширення -- $\Delta _{n} =\Delta _{\max } $ і $\lambda _{n} >\lambda _{n-1} $.

Кожне значення функції $\Sigma $ отримується перерахунком внутрішньої змінної в'язкопружного деформування і розв'язанням внутрішньої задачі. Ця задача полягає у задоволенні закону зчеплення--відриву разом із граничною умовою для розкриття в вершині тріщини:


\begin{equation} \label{GrindEQ__1_17_} 
2(U_{k}^{n-1} +\Delta U_{k}^{n} )=\Delta _{n} , 
\end{equation} 

\title{}\maketitle 

\noindent де $\boldsymbol{\mathrm{U}}^{n-1} $ -- вектор переміщень у вузлах сітки для попереднього моменту часу, $\Delta \boldsymbol{\mathrm{U}}^{n} $ -- приріст переміщень на інтервалі часу $\Delta t_{n} $, індекс $k$ відповідає вертикальному переміщенню вузла сітки, що збігається з вершиною тріщини $\lambda _{n} $. Таким чином, знаходження розв'язку оберненої задачі зводиться до послідовного розв'язання рівнянь


\begin{equation} \label{GrindEQ__1_18_} 
\Sigma (\Delta t_{n} ,\Delta _{n} ,\lambda _{n} )=p,\quad n=1,2,\ldots  
\end{equation} 

\title{}\maketitle 

\noindent відносно невідомого приросту часу. В результаті отримаємо координату положення вершини фізичної тріщини в залежності від часу.

Функція $\Delta $ є зростаючою відносно $\Delta t_{n} $. Якщо для $\Delta t_{n} $ близьких до нуля $\Sigma >p$, то починаючи з $t_{n} $ тріщина переходить на динамічний режим поширення. 

Таким чином, покладаючи в основу обернену схему розв'язання задачі про розкриття тріщини, шляхом розбиття задачі на зовнішню (відносно часу) і внутрішню (відносно закону зчеплення--відриву) сформульовано алгоритм дослідження і поширення тріщини з концентратора напруження в елементі конструкції зі спадкового матеріалу.

\eject \textbf{2 МОДЕЛЮВАННЯ ПОВІЛЬНОГО РОСТУ ТРІЩИНИ У ЛІНІЙНО В'ЯЗКОПРУЖНІЙ ПЛАСТИНІ ІЗ НАЯВНОЮ ТРІЩИНОЮ АБО КОНЦЕНТРАТОРОМ НАПРУЖЕННЯ}

\noindent 
\title{}\maketitle 

\textbf{2.1 Розв'язок задачі поширення наявної до прикладання навантаження тріщини }

\textbf{}

Для апробації алгоритму представленому в попередньому розділі проведемо порівняння розв'язку задачі з аналогічним розв'язком для нескінченої пластини, який базується на відомому аналітичному виразі для розкриття тріщини і більш простому алгоритмі дослідження квазістатичного зростання тріщини.

Розглянемо тріщину нормального відриву у нескінченній площині, яку моделюватиме в рамках моделі зони зчеплення (Рис. 2.1).

\includegraphics*[width=2.83in, height=2.11in]{image1}

Рисунок 2.1 -- Тріщина нормального відриву в нескінченній пластині з ілюстрацією параметрів її моделювання



В'язкоупружне розкриття отримаємо на основі відповідного пружного розв'язку задачі та принципу пружно--в'язкопружної аналогії.

Нехай зовнішнє навантаження прикладене в момент часу $t=0$. Вираз для відриву в точці $x$ на осі $x_{1} $, що відповідає моменту часу $t$, запишемо у формі конститутивних співвідношень спадкової теорії пружності:


\begin{equation} \label{GrindEQ__2_1_} 
\Delta (t,x)=l(t)\tilde{\Delta }[x,\lambda (0)]+\int _{0}^{t} l(t-\tau )\tilde{\Delta }'_{\tau } [x,\lambda (\tau )]{\rm d}\tau . 
\end{equation} 



З огляду на те, що під час інкубаційного періоду, який триває до моменту часу $t=t_{0} $, напівдовжина тріщини $\lambda $ не змінюється, $\tilde{\Delta }[x,\lambda (0)]=\tilde{\Delta }[x,\lambda (t_{0} )]$. 

Вираз \eqref{GrindEQ__2_1_} для розкриття у вершині тріщини, що зростає, має вигляд


\begin{equation} \label{GrindEQ__2_2_} 
\Delta [t,\lambda (t)]=l(t)\tilde{\Delta }[\lambda (t),\lambda (t_{0} )]+\int _{t_{0} }^{t} l(t-\tau )\tilde{\Delta }'_{\tau } [\lambda (t),\lambda (\tau )]{\rm d}\tau . 
\end{equation} 



Будемо шукати положення вершини $\lambda $ у моменти часу $t_{k} =k\cdot \Delta t$, $k=1,2,\ldots $. Позначимо $\lambda _{k} =\lambda (t_{k} )$ та прирівняємо вираз для розкриття \eqref{GrindEQ__2_2_} критичному значенню, отримаємо рівняння для визначення $\lambda _{k} $:


\begin{equation} \label{GrindEQ__2_3_} 
\begin{array}{c} {l(t_{k} )D_{0} +\sum _{i=1}^{k} \Lambda _{i} (D_{i} -D_{i-1} )=\Delta _{\max } ,} \\ {\Lambda _{i} =\frac{1}{\Delta t} \int _{t_{i-1} }^{t_{i} } l(t_{k} -\tau ){\rm d}\tau ,\quad D_{i} =\tilde{\Delta }(\lambda _{k} ,\lambda _{i} ).} \end{array} 
\end{equation} 

\title{}\maketitle 

Геометричні характеристики $D$ і $\tilde{\Delta }$ ілюструє Рис. 2.2.

\noindent 
\title{\eject }\maketitle 

\noindent ----\includegraphics*[width=3.34in, height=1.91in]{image2}

У випадку нескінченної пластини, розкриття тріщини напівдовжини $\lambdaup$ в точці x ($-\beta \le x\le \beta $, $\beta =\lambda \sec (\pi p/2\sigma _{\max } )$) 


\[\widetilde{\Delta }(x,\lambda )=\frac{L}{\pi } \sigma K(x),\; \; \; \; K(x)=K(x,\lambda )-K(x,-\lambda ), \] 
\eqref{GrindEQ__2_4_}
\[K(x,\xi )=(x-\xi )C(x,\xi ),\; \; \; C(x,\xi )=\ln \left|\frac{\widehat{X}(x)-\widehat{X}(\xi )}{\widehat{X}(x)+\widehat{X}(\xi )} \right|,\; \; \; \; \widehat{X}(x)=\sqrt{\frac{x+\lambda }{\lambda -x} } . \] 

\title{}\maketitle 

Отже, рівняння \eqref{GrindEQ__2_3_} дозволяє послідовно визначати точки кінетичної кривої $\lambda _{k} $. Тривалість інкубаційного періоду $t_{0} $ знайдемо з рівняння


\begin{equation} \label{GrindEQ__2_5_} 
l(t_{0} )D_{0} =\Delta _{\max } ,\quad D_{0} =\tilde{\Delta }(\lambda _{0} ,\lambda _{0} ). 
\end{equation} 



Якщо характеристика повзучості $l(t)$ знайдена у формі


\begin{equation} \label{GrindEQ__2_6_} 
\begin{array}{ll} {l(t)} & {=1+\sum _{r} \gamma _{r} \int _{0}^{t} \exp (-\eta _{r} \tau ){\rm d}\tau =} \\ {} & {=l_{\infty } -\sum _{r} \xi _{r} \exp (-\eta _{r} t),\quad \xi _{r} =\frac{\gamma _{r} }{\eta _{r} } ,\quad l_{\infty } =1+\sum _{r} \xi _{r} ,} \end{array} 
\end{equation} 

то при знаходженні $\lambda _{k} $ будемо визначати


\begin{equation} \label{GrindEQ__2_7_} 
\Lambda _{i} =l_{\infty } -\sum _{r} \frac{\xi _{r} \exp (-\eta _{r} (k-i)\cdot \Delta t)\left[1-\exp (-\eta _{r} \cdot \Delta t)\right]}{\eta _{r} \cdot \Delta t} . 
\end{equation} 



Коли $r=1$


\begin{equation} \label{GrindEQ__2_8_} 
l(t)=l_{\infty } -(l_{\infty } -1)\exp (-\eta t). 
\end{equation} 



В цьому випадку тривалість інкубаційного періоду визначається з \eqref{GrindEQ__2_5_}:


\begin{equation} \label{GrindEQ__2_9_} 
t_{0} =\eta ^{-1} \ln \frac{l_{\infty } -1}{l_{\infty } -\Delta _{\max } /D_{0} } . 
\end{equation} 



Умова існування періоду докритичного поширення тріщини


\begin{equation} \label{GrindEQ__2_10_} 
l_{\infty }^{-1} <\frac{D_{0} }{\Delta _{\max } } <1. 
\end{equation} 



Нерівність $D_{0} /\Delta _{\max } \mathrm{\} }l_{\infty }^{-1} $ забезпечує неможливість досягнення розкриттям у вершині тріщини свого граничного значення впродовж скільки завгодно тривалого проміжку часу, нерівність $D_{0} /\Delta _{\max } \mathrm{\sim }1$ відповідає початку динамічного зростання тріщини в момент прикладання навантаження.

\eject \textbf{2.2 Числовий приклад дослідження збіжності запропонованого алгоритму}



Проведемо порівняння розв'язку для нескінченної пластини з розв'язком для пластини скінченних розмірів (Рис. 2.3). В силу симетрії задачі розв'язок буде побудовано для чверті пластини. На частинах границі $\partial _{1} \Omega $ задано умови симетрії, на \partial _{2} \Omega  -- сили.

\noindent 
\title{\includegraphics*[width=2.54in, height=2.72in]{image7}}\maketitle 

\noindent 
\title{}\maketitle 

\noindent Рисунок 2.3 -- Граничні умови для чверті пластини.

\noindent 
\title{}\maketitle 

Розрахунки проведемо для наступних геометричних параметрів задачі: $a=100$ см, $b=120$ см, $\lambda _{0} =1.5$ см, $\beta =2$ см. Параметри довготривалого деформування: $E^{0} =E^{\infty } +E^{1} =4$ ГПа, $E^{\infty } =1$ ГПа, $\nu =0.3$, $\rho =20$ сек. (ізотропний матеріал). Зазначимо, що для обраного значення характеристики релаксації $\rho $ в'язкопружні модулі набувають своїх граничних значень після двохсотої секунди. Характеристики тріщиностійкості: $\phi =150$ Н/м, $\sigma _{\max } =E/300$. Для заданої елементарної когезійної моделі $\Delta _{\max } =\phi /\sigma _{\max } $. Інтенсивність зовнішнього навантаження виберемо відносно рівня максимальних напружень в тілі: $p=\sigma _{\max } /5$. Мінімальний розмір елемента при розрахунках було пов'язано з довжиною ділянки прикладання когезійних сил $l_{{\rm coh}} =0.5$ см, максимальний розмір $h_{\max } =b/10$, параметр, що контролює швидкість переходу розміру елемента між областями з різною щільністю сітки, вибрано на рівні 1.4. Для дослідження інкубаційного періоду прирости часу знаходились для приросту переміщення у вершині тріщини, що складають десяту частину від $\Delta _{\max } $.

Зауважимо, що в класичному підході до застосування когезійної моделі, закон якої містить ділянку зміцнення, когезійну зону продовжують до границь тіла. Але з огляду на отриманий характер збіжності, такий підхід не є раціональним, бо розбивка включатиме малі елементи за відсутності у відповідній області градієнту напружень. Наявність великих за розміром елементів на шляху поширення тріщини суттєво впливає на швидкість збіжності системи нелінійних рівнянь (внутрішня задача). Тому щільна розбивка була використана лише на невеликому відрізу, який напевне включав когезійну зону протягом часового інтервалу дослідження. Такий підхід вносить незначне збурення напружень в кінці фіктивної тріщини але суттєво економить час розв'язання задачі.

\noindent 
\title{}\maketitle 

\includegraphics*[width=5.81in, height=2.75in]{image10}Рисунок 2.4 -- Порівняння розв'язків для кінетичної кривої розвитку тріщин для нескінченної і скінченної пластин 

\noindent 
\title{}\maketitle 

На Рис. 2.4 наведено кінетичну криву, отриману на основі аналітичного розв'язку для розкриття тріщини і кінетичні криві, отримані з використанням розробленого алгоритму дослідження квазістатичного поширення тріщини. Рисунок ілюструє збіжність розв'язку при збільшенні розбивки вздовж лінії тріщини. Виявилось, що неточність визначення розкриття тріщини суттєво накопичується з часом. Збільшивши густину сітки вздовж лінії поширення тріщини вдалось в чотири рази зменшити відхилення розв'язку від референтної кривої з 2.4\% до 0.6\%.

Далі зупинимось на залежності розв'язку задачі від параметрів ЗЗВ. Визначення сил когезії проведемо для згладженого трапецоїдального ЗЗВ:


\begin{equation} \label{GrindEQ__2_11_} 
\bar{T}(\bar{\Delta })=\left(\begin{array}{cc} {a_{1}^{-1} \bar{\Delta }\left(2-a_{1}^{-1} \bar{\Delta }\right),} & {\bar{\Delta }\in [0,a_{1} )} \\ {1,} & {\bar{\Delta }\in [a_{1} ,a_{2} ]} \\ {(1-\bar{\Delta })^{2} (1+2\bar{\Delta }-3a_{2} )(1-a_{2} )^{-3} ,} & {\bar{\Delta }\in (a_{2} ,1]} \end{array}\right.  
\end{equation} 



\includegraphics*[width=4.30in, height=2.64in]{image11}Після введення сталої $\omega =\int _{0}^{1} \bar{T}(\bar{\Delta }){\rm d}\bar{\Delta }=(3-2a_{1} +3a_{2} )/6$ можна визначити внутрішній параметр моделі $\Delta _{\max } =\phi /(\omega \sigma _{\max } )$, який фігурує в рівняннях квазістатичного поширення тріщини. Стала $\omega \in (0,1)$ є однією з мір відхилення ЗЗВ від рівномірного, який відповідає моделі Дагдейла.



Рисунок 2.5 -- Вплив параметрів форми трапецоїдального закону зчеплення--відриву на кінетику поширення тріщини.



На Рис. 2.5 представлені кінетичні криві зростання тріщини для трьох значень параметра форми $a_{2} $ трапецоїдального ЗЗВ. Цей параметр характеризує положення початку ділянки розміцнення. Нульове значення цього параметру відповідає трикутному закону, одиничне -- рівномірному закону (модель Дагдейла). Криві отримані за тих самих параметрів, що і результати попереднього числового прикладу. Аналізуючи отриманий результат, заначимо, що збільшення ділянки розміцнення (зниження параметра) призводить до зменшення тривалості інкубаційного періоду розвитку тріщини і збільшенню загального часу розповсюдження тріщини. Цей висновок спывпадаэ з результатами [16].



\textbf{2.3 Аналіз відтермінованого руйнування пластини з круглим отвором}

\noindent 
\title{}\maketitle 

Добре відомо, що в околі круглого отвору в нескінченій ізотропній пластині концентрація напруження становить три інтенсивності зовнішнього розтягувального навантаження. Ця концентрація не залежить від пружних, а отже і в'язкопружних, характеристик матеріалу. Якщо рівень зовнішнього навантаження менше третини від максимально можливого напруження в тілі, то відтермінованого руйнування не остерігатиметься. Така сама ситуація і для круглого отвору в скінченній пластині з тією різницею, що рівень концентрації буде більшим. У випадку ортотропії властивостей матеріалу можливе збільшення концентрації напруження з часом, що є особливо небезпечним явищем. Метою даного розділу є дослідження зародження тріщини в точці концентрації напруження на контурі круглого отвору в скінченній ізотропній в'язкопружній пластині (Рис. 2.6).

\noindent 
\title{\eject \includegraphics*[width=5.60in, height=2.77in]{image12}}\maketitle 

Рисунок 2.6 -- Пластина з круглим отвором: геометрія та прикладене навантаження.



Розрахунки проведемо для наступних геометричних параметрів задачі: 100 см, 120 см. Параметри довготривалого деформування:  ГПа, $E^{\infty } =1$ ГПа, $\nu =0.3$, $\rho =20$ сек. (ізотропний матеріал). Характеристики тріщиностійкості і параметри форми ЗЗВ:  Н/м, , a_{1} =a_{2} =0.001. Інтенсивність зовнішнього навантаження виберемо в п'ять разів меншою максимальних напружень в тілі: $p=\sigma _{\max } /5$. Мінімальний розмір елемента при розрахунках було пов'язано з довжиною ділянки прикладання когезійних сил $h_{\min } =l_{{\rm coh}} /100$, l_{{\rm coh}} =0.01R; максимальний розмір елемента сітки h_{\max } =b/10; параметр, що контролює швидкість переходу розміру елемента між ділянками з різною щільністю сітки, вибрано на рівні 1.4. Для дослідження інкубаційного періоду прирости часу знаходились для приросту переміщення у вершині тріщини, що складають соту частину від $\Delta _{\max } $. 

Для заданої конфігурації параметрів при прикладанні зовнішнього навантаження напруження в концентраторі досягає міцності зчеплення при R=47 см. В цей момент в напрямку поширення тріщини в тілі утворюється зона передруйнування, яка моделюється зоною зчеплення. З часом розкриття в зоні зчеплення збільшується, після чого можливе як припинення зародження тріщини так і її квазістатичне або динамічне поширення. 

При збільшені радіусу отвору з 47 до 47.8 см, розкриття берегів фіктивної тріщини припиняється з часом. Це свідчить про існування певного запасу для цілісності конструкції з концентратором напруження при наявності в'язкопружних властивостей. Проілюструємо розв'язок задачі при подальшому збільшенні радіусу отвору.

На Рис. 2.7 для чотирьох радіусів отвору наведено визначені в результаті розв'язання задачі моменти часу, що відповідають досягненню розкриттям заданих долей критичного розкриття в концентраторі. Після останнього зазначеного на кожному з блоків рисунку моменту часу розпочинається динамічне зростання тріщини. Збільшення радіусу отвору зменшує тривалість відтермінованого руйнування. При деякому значенні цього геометричного параметра зароджена тріщина без затримки на відтерміноване руйнування переходитиме на динамічний етап поширення. Звернемо увагу на той факт, що перехід на динамічний етап відбувається до досягнення розкриттям в концентраторі критичного значення -- процес інкубації тріщини не завершено.

\noindent \includegraphics*[width=6.31in, height=6.82in]{image18}

Рисунок 2.7 -- Розкриття в зоні зчеплення, що виходить з точки концентратора напружень біля круглого отвору.



В попередніх числових прикладах було досліджено збіжність алгоритму і встановлено, що щільність сітки вздовж лінії поширення тріщини суттєво впливає на точність розв'язку. У розв'язках, отриманих в поточному параграфі квазістатичне поширення тріщини відсутнє -- відбувається лише зростання довжини зони передруйнування. Зазначимо як впливає величина приросту розкриття в концентраторі на час відтермінованого руйнування. Наприклад, для R=48 см при зменшенні приросту розкриття в два рази тривалість відтермінованого руйнування складає 94.1 сек порівняно з 93.4 сек, зазначених на рисунку 2.7.

\noindent 
\title{}\maketitle 

\textbf{2.4 Аналіз відтермінованого руйнування пластини з ромбовидним отвором}

\textbf{}

На відміну від випадку круглого отвору наявність ромбовидного отвору призводить до нескінченного рівня напружень в концентраторі і, як і у випадку тріщини, зону зчеплення треба вводити для зовнішнього навантаження довільної інтенсивності. Розглянемо постановку аналогічну тій, що була представлена в попередньому параграфі, за винятком форми отвору (Рис. 2.8).

\noindent 
\title{}\maketitle 

\noindent \includegraphics*[width=5.60in, height=3.26in]{image20}Рисунок 2.8 -- Пластина з ромбовидним отвором: геометрія та прикладене навантаження.

\noindent 
\title{}\maketitle 

Числові розв'язки побудуємо для R=1.5 см (такий самий розмір використано для дослідження пластини з наявною при прикладанні навантаження тріщиною). Розв'язки для ізотропного матеріалу побудовано для тих самих механічних характеристик і характеристик тріщиностійкості, що і в попередньому параграфі. Для ортотропного матеріалу обрано наступні параметри: $E_{22}^{0} =E_{22}^{\infty } +E_{22}^{1} =4$ ГПа, $E_{22}^{\infty } =1$ ГПа, $E_{11}^{0} =E_{11}^{\infty } +E_{11}^{1} =4.2$ ГПа, $E_{11}^{\infty } =1$ ГПа, \includegraphics*[width=6.08in, height=5.54in]{image22}$G_{12}^{0} =G_{12}^{\infty } +G_{12}^{1} =2$ ГПа, $G_{12}^{\infty } =0.5$ ГПа, $\nu _{21} =0.3$, $\rho =20$ с.

\noindent 
\title{----}\maketitle 

На Рис. 2.9 для кожного знайденого в рамках запропонованого алгоритму досліджень моменту часу, що відповідає проходженню скінченного елемента вздовж лінії тріщини, наведено розкриття в зоні прикладання когезійних сил. Збільшення миттєвого модуля вздовж лінії тріщини і модуля зсуву призвело до збільшення часу відтермінованого руйнування і проміжку вздовж лінії тріщини, на якому відбувається квазістатичне зростання.

На відміну від результатів попереднього параграфа, відтерміноване руйнування пластини з ромбовидним отвором більшою мірою складається з квазістатичного поширення тріщини, таким чином повторюючи основні тренди для тіла з наявною тріщиною. 

\noindent 
\title{}\maketitle 

\includegraphics*[width=5.88in, height=5.01in]{image24}Рисунок 2.10 -- Зміна напруження в околі ромбовидного отвору в два характерних моменту часу.

\noindent 
\title{}\maketitle 

На Рис. 2.10 проілюстроване поле відносного напруження в околі лінії розповсюдження тріщини. Перший з блоків рисунку відповідає миттєвому розв'язку задачі, другий -- розв'язку в момент, коли розкриття в концентраторі досягає свого критичного значення. З огляду на обраний трикутний ЗЗВ спостерігається значна ділянка деградації когезійних напружень в околі концентратора під час зрушення фізичної тріщини. Лінія, що відповідає лінії тріщини і вздовж якої прикладаються когезійні сили, зображена на рисунку сірим кольором так як і зона зчеплення. Зазначимо, що сітка методу скінченних елементів в рамках запропонованого алгоритму не є адаптивною, тому її треба ущільнити вздовж ліній поширення тріщини. Використання адаптивної сітки в рамках алгоритму ускладнено по причині використання одних і тих самих точок (точок інтегрування) для перерахунку внутрішньої змінної задачі. Для побудови поля напружень коефіцієнт збільшення деформацій дорівнює 20, кількість степенів волі -- 4160.

\eject \textbf{3 ДОСЛІДЖЕННЯ ВІДТЕРМІНОВАНОГО РУЙНУВАННЯ ДЛЯ ПРОСТОРОВОЇ ЗАДАЧІ}

\noindent 
\title{}\maketitle 

\textbf{3.1 Розв'язання просторової задачі відтермінованого руйнування }



Розглянемо задачу, представлену у попередньому розділі, у просторовій постановці. Інтегральний зв'язок між компонентами тензора напружень і деформацій для лінійного в'язкопружного матеріалу, що не старіє, у загальному тривимірному аналізі запишемо в матричній формі


\begin{equation} \label{GrindEQ__3_1_} 
\sigma _{i} (t)=\int _{-\infty }^{t} C_{ik} (t-\tau ){\rm d}\varepsilon _{k} (\tau ), 
\end{equation} 

\title{}\maketitle 

\noindent де $\sigma _{i} $ та $\varepsilon _{k} $ -- компоненти векторів


\begin{equation} \label{GrindEQ__3_2_} 
\boldsymbol{\mathrm{\sigma }}=\{ \sigma _{11} ,\sigma _{22} ,\sigma _{33} ,\sigma _{23} ,\sigma _{31} ,\sigma _{12} \} ^{{\rm T}} ,\quad \boldsymbol{\mathrm{\varepsilon }}=\{ \varepsilon _{11} ,\varepsilon _{22} ,\varepsilon _{33} ,\varepsilon _{23} ,\varepsilon _{31} ,\varepsilon _{12} \} ^{{\rm T}} . 
\end{equation} 



Імплементації в'язкопружної моделі в розрахункову схему метода скінченних елементів проводиться у повній аналогії з алгоритмом, представленим для плоскої задачі.

Матриця жорсткості для задачі теорії пружності має наступний вигляд:


\begin{equation} \label{GrindEQ__3_3_} 
{\rm D}=\left[\begin{array}{cccccc} {\frac{1-\nu _{21} \nu _{32} }{E_{22} E_{33} \Delta } } & {\frac{\nu _{21} +\nu _{32} \nu _{23} }{E_{22} E_{33} \Delta } } & {\frac{\nu _{32} +\nu _{21} \nu _{32} }{E_{22} E_{33} \Delta } } & {0} & {0} & {0} \\ {} & {\frac{1-\nu _{21} \nu _{32} }{E_{22} E_{33} \Delta } } & {\frac{\nu _{32} +\nu _{21} \nu _{32} }{E_{22} E_{33} \Delta } } & {0} & {0} & {0} \\ {} & {} & {\frac{1-\nu _{21}^{2} }{E_{22}^{2} \Delta } } & {0} & {0} & {0} \\ {} & {{\kern 1pt} A8<5B@VO{\kern 1pt} } & {} & {G_{23} } & {0} & {0} \\ {} & {} & {} & {} & {G_{23} } & {0} \\ {} & {} & {} & {} & {} & {\frac{E_{22} }{2(1+\nu _{21} )} } \end{array}\right], 
\end{equation} 


\begin{equation} \label{GrindEQ__3_4_} 
\quad \nu _{23} E_{33} =\nu _{32} E_{22} ,\quad \Delta =\frac{(1+\nu _{21} )(1-\nu _{21} -2\nu _{23} \nu _{32} )}{E_{22}^{2} E_{33} } . 
\end{equation} 



Побудуємо в'язкопружний аналог цієї матриці для випадку, коли кожен з модулів $E_{33} $, $E_{22} $ та $G_{23} $ задано однією експоненціальною функцією з тим самим часом релаксації:


\[E_{33} (t)=E_{33}^{\infty } +E_{33}^{1} \exp \{ -t/\rho \} ,\quad E_{22} (t)=E_{22}^{\infty } +E_{22}^{1} \exp \{ -t/\rho \} ,\] 
\begin{equation} \label{GrindEQ__3_5_} 
G_{23} (t)=G_{23}^{\infty } +G_{23}^{1} \exp \{ -t/\rho \} ,\quad \nu _{21} ={\rm const},\quad \nu _{32} ={\rm const}. 
\end{equation} 


В цьому випадку легко знайти в'язкопружний аналог матриці $\boldsymbol{\mathrm{D}}=[D_{ij} ]$. Для цього підставимо в \eqref{GrindEQ__3_5_} перетворення Лапласа--Карсона відповідних модулів. Обернене перетворення дає


\[D_{11} (t)=\frac{b_{\infty } }{a_{\infty } } +\frac{b_{1} }{a_{1} } \exp \{ -t/\rho \} -c\exp \{ -t/\rho _{2} \} ,\] 
\[D_{12} (t)=\frac{d_{\infty } }{a_{\infty } } +\frac{d_{1} }{a_{1} } \exp \{ -t/\rho \} -c\exp \{ -t/\rho _{2} \} ,\] 
\[D_{13} (t)=\nu _{32} \left[\frac{E_{22}^{\infty } E_{33}^{\infty } }{a_{\infty } } +\frac{E_{22}^{1} E_{33}^{1} }{a_{1} } \exp \{ -t/\rho \} -2c\exp \{ -t/\rho _{2} \} \right],\] 
\[D_{33} (t)=(1-\nu _{21} )\left[\frac{(E_{33}^{\infty } )^{2} }{a_{\infty } } +\frac{(E_{33}^{1} )^{2} }{a_{1} } \exp \{ -t/\rho \} \right]-4c\nu _{32}^{2} \exp \{ -t/\rho _{2} \} ,\] 
\begin{equation} \label{GrindEQ__3_6_} 
D_{44} (t)=D_{55} (t)=G_{23} (t),\quad D_{66} (t)=\frac{E_{22} (t)}{2(1+\nu _{21} )} , 
\end{equation} 



\noindent де
\[\rho _{2} =\frac{a_{0} }{a_{\infty } } \rho ,\quad a_{i} =E_{33}^{i} (1-\nu _{21} )-2\nu _{32}^{2} E_{22}^{i} ,\quad b_{i} =\frac{E_{33}^{i} -\nu _{32}^{2} E_{22}^{i} }{1+\nu _{21} } E_{22}^{i} ,\] 
\begin{equation} \label{GrindEQ__3_7_} 
c=\frac{(1-\nu _{21} )\nu _{32}^{2} (E_{22}^{\infty } E_{33}^{0} -E_{22}^{0} E_{33}^{\infty } )^{2} }{a_{0} a_{1} a_{\infty } } ,\quad d_{i} =\frac{\nu _{21} E_{33}^{i} +\nu _{32}^{2} E_{22}^{i} }{1+\nu _{21} } E_{22}^{i} , 
\end{equation} 
\[i=0,1,\infty .\] 

\title{}\maketitle 


\title{Таким чином, наведено необхідні вихідні данні для інкременталізації конститутивних співвідношень. За допомогою запропонованого алгоритму дослідимо відтерміноване руйнування пластини і циліндричної оболонки з отворами. }\maketitle 

\noindent 
\title{}\maketitle 

\textbf{3.2 Числовий приклад розрахунку відтермінованого руйнування пластини із концентратором напружень}

\noindent 
\title{}\maketitle 

Числові розв'язки проілюструємо для ізотропного матеріалу: $E_{33}^{0} =E_{33}^{\infty } +E_{33}^{1} =4$ ГПа, $E_{33}^{\infty } =1$ ГПа, $E_{22}^{0} =E_{22}^{\infty } +E_{22}^{1} =4$ ГПа, $E_{22}^{\infty } =1$ ГПа, $G_{23}^{0} =G_{23}^{\infty } +G_{23}^{1} =E_{33}^{0} /2(1+\nu )$, $G_{23}^{\infty } =E_{33}^{\infty } /2(1+\nu )$, $\nu _{21} =\nu _{32} =\nu =0.3$, $\rho =20$ сек. В силу симетрії будемо розглядати восьму частину пластини, геометрія якої представлена на Рис. 3.1. Напівдовжина і напівширина пластини $a=10$ см, напівдіагональ квадрату$R=1.5$ см. 

\noindent 
\title{}\maketitle 

\noindent 
\title{}\maketitle 

\noindent 
\title{\includegraphics*[width=2.20in, height=2.86in]{image25}}\maketitle 

\noindent 
\title{\includegraphics*[width=4.17in, height=2.84in]{image26}}\maketitle 

\noindent 
\title{}\maketitle 

\noindent 
\title{}\maketitle 

\noindent 
\title{}\maketitle 

\noindent 
\title{}\maketitle 

\noindent 
\title{}\maketitle 

\noindent 
\title{}\maketitle 

\noindent 
\title{}\maketitle 

\noindent 
\title{}\maketitle 

\noindent 
\title{}\maketitle 

\noindent 
\title{}\maketitle 

\noindent 
\title{}\maketitle 

\noindent 
\title{}\maketitle 

\noindent 
\title{}\maketitle 

\noindent 
\title{}\maketitle 

\noindent 
\title{}\maketitle 

\noindent 
\title{}\maketitle 

\noindent 
\title{----}\maketitle 

На Рис. 3.2 проілюстровано розв'язок задачі для дванадцяти моментів часу визначених в рамках запропонованого алгоритму. Сині криві для розкриття відповідають інкубаційному періоду зростання тріщини, червоні -- періоду квазістатичного зростання. Характеристики тріщиностійкості і параметри форми ЗЗВ:  Н/м, \sigma _{\max } =E_{33} /1000, a_{1} =a_{2} =0.001. Інтенсивність зовнішнього навантаження -- $p=0.45\sigma _{\max } $. Мінімальний розмір елемента при розрахунках h_{\min } =0.1 см; максимальний розмір елемента сітки h_{\max } =a/5; параметр, що контролює швидкість переходу розміру елемента між областями з різною щільністю сітки, вибрано на рівні 1.35. Для дослідження інкубаційного періоду прирости часу знаходились для приросту переміщення у вершині тріщини, що складають п'яту частину від $\Delta _{\max } $.

\includegraphics*[width=5.10in, height=2.34in]{image31}

Рисунок 3.2 -- Відносне розкриття тріщини та кінетична крива її зростання.\textit{}

\noindent 

Наслідком того, що було обрано міцність зчеплення більше ніж в три рази меншою за ту, що була використана при розрахунках в плоскій задачі, стало суттєве збільшення довжини когезійної зони. 

На Рис. 3.3 представлені розподіли відносного напруження \sigma _{33} /\sigma _{\max }  в кожний з моментів часу, для яких було знайдено розв'язок.

\noindent 

\noindent \includegraphics*[width=6.68in, height=7.24in]{image33}



Рисунок 3.3 -- Деформована пластина і поля напружень для певних моментів часу.

\noindent 


\title{На Рис. 3.3 проілюстровано деформування пластини (із фактором 100) і еволюція поля напружень з часом. Представлені дані повторюють тренди для аналогічної пластини у двовимірній постановці, яка розглянута у попередньому розділі, що є свідченням адекватності побудованого методу та розрахункового алгоритму.}\maketitle 


\title{}\maketitle 

\textbf{3.3 Числовий приклад розрахунку відтермінованого руйнування циліндричної оболонки із концентратором напружень }

\noindent 
\title{}\maketitle 

Числові розв'язки проілюструємо для ізотропного матеріалу: $E_{33}^{0} =E_{33}^{\infty } +E_{33}^{1} =4$ ГПа, $E_{33}^{\infty } =1$ ГПа, $E_{22}^{0} =E_{22}^{\infty } +E_{22}^{1} =4$ ГПа, $E_{22}^{\infty } =1$ ГПа, $G_{23}^{0} =G_{23}^{\infty } +G_{23}^{1} =E_{33}^{0} /2(1+\nu )$, $G_{23}^{\infty } =E_{33}^{\infty } /2(1+\nu )$, $\nu _{21} =\nu _{32} =\nu =0.3$, $\rho =20$ сек. В силу симетрії будемо розглядати восьму частину циліндра, геометрія якого представлена на Рис. 3.4. Довжина циліндра складає 20 см., зовнішній радіус 5 см., товщина стінки циліндра 0.4 см., напівдіагональ отвору$R=1.5$ см. 

\noindent 
\title{}\maketitle 

\noindent 
\title{\includegraphics*[width=3.48in, height=2.86in]{image34}}\maketitle 

Рисунок 3.4 -- Розрахункова модель циліндра.

\noindent 

На Рис. 3.5 проілюстровано розв'язок задачі для восьми моментів часу визначених в рамках запропонованого алгоритму. На першому блоці рисунка наведені залежності відносного розкриття від довжини дуги $s$ вздовж перерізу циліндра від концентратора напружень. Сині криві для розкриття відповідають інкубаційному періоду зростання тріщини, червоні -- періоду квазістатичного зростання. Характеристики тріщиностійкості і параметри форми ЗЗВ:  Н/м, \sigma _{\max } =E_{33} /1000, a_{1} =a_{2} =0.001. Інтенсивність зовнішнього навантаження виберемо на рівні $p=0.3\sigma _{\max } $. Мінімальний розмір елемента при розрахунках h_{\min } =0.1 см; максимальний розмір елемента сітки h_{\max } =a/5; параметр, що контролює швидкість переходу розміру елемента між областями з різною щільністю сітки, вибрано на рівні 1.35. Для дослідження інкубаційного періоду прирости часу знаходились для приросту переміщення у вершині тріщини, що складають п'яту частину від $\Delta _{\max } $.

\noindent 
\title{\includegraphics*[width=5.23in, height=2.53in]{image39}}\maketitle 

\noindent 
\title{}\maketitle 

Рисунок 3.5 -- Відносне розкриття тріщини та відповідна кінетична крива зростання.\textit{}

\noindent 
\title{}\maketitle 

На Рис. 3.6 проілюстровано розв'язок задачі для певного моменту часу.

\noindent 
\title{}\maketitle 

\includegraphics*[width=6.07in, height=2.80in]{image40}Рисунок 3.6 -- Деформований циліндр (з фактором 100) і поле відносного напруження $\sigma _{33} /\sigma _{\max } $ для моменту закінчення інкубації тріщини.\textit{}

\noindent 
\title{}\maketitle 


\title{Якщо тренди для просторової пластини схожі на тренди плоскої задачі з незначним збільшенням когезійної довжини в середині пластини, то для оболонки картина суттєво відрізняється: на внутрішній стороні оболонки когезійна довжина суттєво збільшилась порівняно із її значенням на зовнішній стороні.}\maketitle 

\noindent 
\title{}\maketitle 

\eject \textbf{ВИСНОВКИ}

\noindent 
\title{}\maketitle 

Розроблено метод моделювання відтермінованого руйнування в'язкопружних елементів конструкцій з концентраторами напружень. Метод побудовано на основі інтеграцїї підходів до дослідження змінного з часом напружено-деформованого стану із моделями розвитку тріщин у в'язкопружних елементах конструкцій.

Результати застосування методу порівняно з розв'язком для канонічної області, для якої у випадку моделі Дагдейла існує аналітичний вираз для розкриття тріщини, і апробовано на розв'язанні задач визначення напружено-деформівного стану пластин із різними конфігураціями концентраторів напружень за умов докритичного та критичного станів.

Проаналізовано повільне зростання тріщин, яке обумовлене спадковими в'язкопружними властивостями матеріалу. Визначені моменти часу, що відповідають досягненню критичного розкриття в концентраторі та момент початку динамічного зростання тріщин.

Для пластин із круговим отвором встановлено, що збільшення радіусу отвору зменшує тривалість відтермінованого руйнування. При деякому значенні радіуса концентратора зароджена тріщина без затримки на відтерміноване руйнування переходить на динамічний етап поширення, при чому перехід на динамічний етап відбувається до досягнення розкриттям в концентраторі критичного значення -- процес інкубації тріщини не завершено.

Відтерміноване руйнування пластини з ромбовидним отвором більшою мірою складається з квазістатичного поширення тріщини, таким чином повторюючи основні тренди для тіла з наявною тріщиною. Для оболонки картина суттєво відрізняється: на внутрішній її стороні когезійна довжина суттєво збільшилась порівняно із її значенням на зовнішній стороні. Це спостереження підкреслює вплив геометричних факторів і структурної конфігурації на часову еволюцію відтермінованого руйнування в'язкопружних систем.\eject \textbf{ПЕРЕЛІК ДЖЕРЕЛ ПОСИЛАННЯ}

\noindent 
\title{}\maketitle 

\begin{enumerate}
\item  Hui C.-Y., Zhu B., Long R. Steady state crack growth in viscoelastic solids: A comparative study. \textit{J. Mech. Phys. Solids}. 2022. 159, P. 104748.

\item  Ricks E. An incremental approach to the solution of snapping and buckling problems. \textit{Int. J. Solids Structures.} 1979. 15, P. 524$\mathrm{-}$551.

\item  Crisfield M.A. A fast incremental/iterative solution procedure that handles ``snap-through''. \textit{Comput. Struct}. 1981. 13. No 1--3, P. 55$\mathrm{-}$62.

\item  Gao Y.F., Bower A.F. A simple technique for avoiding convergence problems in finite element simulations of crack nucleation and growth on cohesive interfaces. \textit{Modelling Simul. Mater. Sci. Eng}. 2004. 12, P. 453$\mathrm{-}$463.

\item  Brockway G.S., Schapery R.A. Some viscoelastic crack growth relations for orthotropic and prestrained media. \textit{Eng. Fract. Mech}. 1978. No 10. P. 453$\mathrm{-}$468. 

\item  Dugdale D.S. Yielding of steel sheets containing slits. \textit{J. Mech. Phys. Solids}. 1960. No 8. P. 100$\mathrm{-}$104. 

\item  Barenblatt G.I. The mathematical theory of equilibrium cracks in brittle fracture. \textit{Adv. Appl. Mech.} 1962. No 7. P. 55$\mathrm{-}$129.

\item  Hillerborg A., Modeer M., Petersson P.E. Analysis of crack formation and crack growth in concrete by means of fracture mechanics and finite elements. \textit{Cem. Concr. Res.} 1976. No 6. P. 773$\mathrm{-}$781. 

\item  Zobeiry N. та ін. A differential approach to finite element modelling of isotropic and transversely isotropic viscoelastic materials. J. \textit{Mechanics of Materials.} 2016. \textbf{97}. P. 76$\mathrm{-}$91. 

\item  Zocher M.A., Groves S.E., Allen D.H. A three dimensional finite element formulation for thermoviscoelastic orthotropic media. Int. J. \textit{Numer. Meth. Engng. }1997. \textbf{40}. № 12. P. 2267$\mathrm{-}$2288.

\item  Селіванов М.Ф., Фернаті П.В. Дослідження зміни концентрації напружень у просторовій пластині з в'язкопружного трансверсально ізотропного матеріалу. Допов. Нац. акад. наук Укр. 2023. No 1. С. 33---39. https://doi.org/10.15407/dopovidi2023.01.033

\item  Селіванов М.Ф., Фернаті П.В. Моделювання квазістатичного поширення тріщини у в'язкопружному ортотропному середовищі в рамках підходу інкременталізації конститутивних рівнянь. Допов. Нац. акад. наук Укр. 2023. No 2. С. 65---75. https://doi.org/10.15407/dopovidi2023.02.065

\item  Селіванов М.Ф., Фернаті П.В. Ініціація і повільне поширення тріщини вздовж площини симетрії просторової в'язкопружної трансверсально ізотропної пластини. Допов. Нац. акад. наук Укр. 2023. № 4. С. 26---32. https://doi.org/10.15407/dopovidi2023.04.026

\item  Селіванов М.Ф., Фернаті П.В., Терещенко Л.М. Визначення траєкторії розвитку тріщини з використанням адаптивної розбивки із регулярною частиною // Матеріали міжнародної наукової конференції «Механіка: сучасність і перспективи -- 2024» -- С. 254-256.

\item  Селіванов М.Ф., Чорноіван Ю.О. Дослідження траєкторії тріщини змішаного режиму руйнування за допомогою неявної схеми. Допов. Нац. акад. наук Укр. 2024, №~6. https://doi.org/10.15407/dopovidi2024.06

\item  Селіванов М.Ф., Фернаті П.В. Вплив параметрів когезійного закону на критичне навантаження тіла з тріщиною нормального відриву // «Актуальні проблеми механіки -- 2023» Київ 14--16 листопада 2023.
\end{enumerate}

\noindent 
\title{}\maketitle 


\end{document}

