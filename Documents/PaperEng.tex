\documentclass[11pt,a4paper]{article}

% =====================================================
% Encoding and language
% =====================================================
\usepackage[T1]{fontenc}
\usepackage[utf8]{inputenc}
\usepackage[english]{babel}

% =====================================================
% Page layout
% =====================================================
\usepackage[
left=30mm,
right=30mm,
top=30mm,
bottom=30mm
]{geometry}

% =====================================================
% Mathematics and symbols
% =====================================================
\usepackage{amsmath,amssymb,amsfonts}
\usepackage{bm}
\usepackage{mathtools}

% =====================================================
% Figures and tables
% =====================================================
\usepackage{graphicx}
\usepackage{caption}
\usepackage{subcaption}

% =====================================================
% References
% =====================================================
\usepackage[numbers]{natbib}

% =====================================================
% Document
% =====================================================
\title{Algorithmic Framework for Quasi-Static Crack Growth\\
	in Viscoelastic Transversely Isotropic Media}
\author{Mikhailo Selivanov}
\date{\today}

\begin{document}
	\maketitle
	
	\begin{abstract}
		This document presents a baseline formulation of a numerical framework for
		modeling delayed fracture and quasi-static crack growth in viscoelastic
		transversely isotropic solids using a cohesive zone approach and incremental
		constitutive equations.
		
		The constitutive modeling, fracture criteria, and finite element formulation
		are described in a consistent manner. However, the algorithm for determining
		time increments during crack incubation and propagation is acknowledged to be
		provisional. In particular, for certain load and crack configurations, the time
		increment could not be reliably determined, which may affect quantitative
		conclusions drawn from earlier numerical results.
		
		The present document therefore serves as a reference formulation for further
		algorithmic refinement and code development rather than as a source of validated
		numerical predictions.
	\end{abstract}
	
	\section{Scope and limitations of the present formulation}
	
	The purpose of this document is to fix the mechanical model, notation, and
	overall algorithmic intent before revising the numerical procedure for time
	integration. While the viscoelastic constitutive relations and the cohesive-zone
	fracture framework are considered reliable, the time-stepping strategy used in
	earlier computations requires reassessment.
	
	Consequently, numerical results reported previously should be interpreted as
	illustrative only and not as quantitatively validated predictions.
	
	\section{Problem statement}
	
	We consider a solid body containing a pre-existing crack subjected to
	quasi-static loading. The material outside the fracture process zone is modeled
	as a linear viscoelastic transversely isotropic medium. Crack initiation and
	propagation are described using a cohesive zone model with a prescribed
	traction--separation law.
	
	The analysis is restricted to small deformations and isothermal conditions.
	Body forces are neglected.
	
	\section{Viscoelastic constitutive formulation}
	
	The stress--strain relationship is written in the Boltzmann--Volterra form
	\begin{equation}
		\sigma_{ij}(t) =
		\int_{-\infty}^{t}
		C_{ijkl}(t-\tau)\,
		\frac{\partial \varepsilon_{kl}(\tau)}{\partial \tau}
		\,\mathrm{d}\tau ,
	\end{equation}
	where $C_{ijkl}(t)$ are the relaxation functions of a non-aging viscoelastic
	transversely isotropic material.
	
	For numerical implementation, the constitutive equations are rewritten in
	incremental form using internal variables, allowing efficient evaluation within
	a finite element framework.
	
	\subsection{Incremental quasi-static formulation}
	\label{sec:incremental_qs}
	
	The evolution of the crack and the associated viscoelastic response is modeled
	as a sequence of quasi-static increments. Each increment connects two
	equilibrium states and is characterized by an unknown time duration
	$\Delta t>0$, which is determined as part of the solution rather than prescribed
	\emph{a priori}.
	
	At the beginning of an increment, the current state of the system is assumed
	known, including the displacement field, the internal viscoelastic history
	variables, and the position of the active cohesive zone. Within the increment,
	the response is governed by a viscoelastic constitutive law and a nonlinear
	cohesive traction--separation relation.
	
	\paragraph{Incremental equilibrium problem.}
	For a given trial value of $\Delta t$, the viscoelastic constitutive update
	yields an effective tangent stiffness operator $\mathbf{K}(\Delta t)$ and a
	history-dependent residual force vector $\mathbf{F}^{t\sigma}(\Delta t)$.
	The incremental equilibrium problem is then formulated as a coupled system for
	the displacement increment $\Delta\mathbf{u}$ and a scalar load parameter
	$\Sigma$:
	\begin{equation}
		\mathbf{K}(\Delta t)\,\Delta\mathbf{u}
		-
		\left(
		\Sigma\,\mathbf{r}_\mathrm{ext}
		+
		\mathbf{r}_\mathrm{c}(\Delta\mathbf{u})
		-
		\mathbf{F}^{p\sigma}
		+
		\mathbf{F}^{t\sigma}(\Delta t)
		\right)
		= \mathbf{0},
		\label{eq:incremental_equilibrium}
	\end{equation}
	where $\mathbf{r}_\mathrm{ext}$ denotes the external load vector,
	$\mathbf{r}_\mathrm{c}$ is the cohesive force vector, and
	$\mathbf{F}^{p\sigma}$ represents the contribution of stresses accumulated in
	previous increments.
	
	To ensure a unique solution of the incremental problem, an additional
	kinematic constraint is imposed. In the present work, the constraint prescribes
	the opening displacement at a selected cohesive index $c_i$,
	\begin{equation}
		u_{c_i} + \Delta u_{c_i} = \frac{k_D D_\mathrm{ym}}{2},
		\label{eq:opening_constraint}
	\end{equation}
	where $k_D$ is a dimensionless control parameter and $D_\mathrm{ym}$ is a
	characteristic opening scale.
	
	Equations~\eqref{eq:incremental_equilibrium} and
	\eqref{eq:opening_constraint} form a coupled nonlinear system for
	$(\Delta\mathbf{u},\Sigma)$, which is solved quasi-statically for each trial
	value of $\Delta t$.
	
	\paragraph{Continuation condition for the time increment.}
	For a fixed internal state and control parameter $k_D$, the solution of the
	incremental problem yields a scalar quantity $\Sigma(\Delta t)$, representing
	the external load required to satisfy equilibrium and the imposed opening
	constraint over an increment of duration $\Delta t$.
	
	The admissible time increment is defined by the scalar continuation condition
	\begin{equation}
		\Sigma(\Delta t) = \Sigma_{\mathrm{ext}},
		\label{eq:dt_condition}
	\end{equation}
	where $\Sigma_{\mathrm{ext}}$ is the prescribed external load level. The
	existence of a quasi-static increment therefore reduces to the existence of a
	root of Eq.~\eqref{eq:dt_condition} for $\Delta t>0$.
	
	\paragraph{Physical interpretation.}
	In the adopted viscoelastic formulation, increasing $\Delta t$ enhances stress
	relaxation and reduces the effective stiffness of the structure. As a result,
	the function $\Sigma(\Delta t)$ is expected to decrease with increasing
	$\Delta t$ along a continuously tracked equilibrium branch. The limiting case
	$\Delta t\to 0^{+}$ corresponds to an instantaneous elastic response, while
	large values of $\Delta t$ approach the fully relaxed regime.
	
	If Eq.~\eqref{eq:dt_condition} admits no solution for $\Delta t>0$, no
	quasi-static increment exists that satisfies the prescribed loading and
	constraint. This criterion provides a physically meaningful distinction between
	the non-existence of quasi-static solutions and purely numerical convergence
	issues.
	
	
	\section{Cohesive zone model}
	
	Crack growth is modeled by a cohesive zone located ahead of the crack tip.
	The normal cohesive traction is assumed to depend on the local crack opening
	displacement $\Delta$ via a prescribed traction--separation law
	\begin{equation}
		T = T(\Delta),
	\end{equation}
	which is assumed to be non-negative and vanishing at a critical opening
	$\Delta_{\max}$.
	
	Crack propagation is governed by the condition
	\begin{equation}
		\Delta(\lambda,t) = \Delta_{\max},
	\end{equation}
	where $\lambda$ denotes the current crack length.
	
	\section{Algorithmic structure}
	
	The computational procedure conceptually consists of the following stages:
	\begin{enumerate}
		\item instantaneous elastic response at load application;
		\item crack incubation under viscoelastic deformation at fixed crack length;
		\item quasi-static crack growth under sustained loading.
	\end{enumerate}
	
	At each stage, the displacement field is obtained by enforcing equilibrium and
	the cohesive traction--separation law. Time increments are determined by solving
	a nonlinear auxiliary problem that links the applied load, crack geometry, and
	crack opening.
	
	\section{Current status of the time-increment procedure}
	
	The determination of time increments represents the most sensitive part of the
	algorithm. In its current form, the procedure may fail to identify a valid time
	increment for certain steps of crack incubation or propagation. This deficiency
	directly affects the reliability of time-dependent results, such as predicted
	incubation times and crack growth histories.
	
	For this reason, the time-integration strategy must be revised and validated
	before any quantitative conclusions can be drawn.
	
	\subsection{Solvability of the time-increment equation}
	\label{sec:solvability_dt}
	
	The key computational step of the proposed framework is the determination of the
	time increment $\Delta t_n$ from the scalar nonlinear equation
	\begin{equation}
		\Sigma(\Delta t_n,\Delta_n,\lambda_n) = \sigma^{(\mathrm{ext})},
		\label{eq:dt_equation}
	\end{equation}
	where $\Sigma(\Delta t,\Delta,\lambda)$ denotes the external load level required to
	achieve (after the viscoelastic increment of duration $\Delta t$) the prescribed
	crack-tip opening $\Delta$ for a prescribed crack geometry characterized by
	$\lambda$. Equation~\eqref{eq:dt_equation} is the natural coupling condition between
	(i) the incremental viscoelastic update and (ii) the fracture constraint
	$\Delta(\lambda_n,t_n)=\Delta_n$ imposed at the crack tip while satisfying the
	cohesive traction--separation law along the cohesive segment.
	
	\paragraph{Expected qualitative properties.}
	For fixed $(\Delta_n,\lambda_n)$, the function $\Sigma(\Delta t,\Delta_n,\lambda_n)$
	is expected to be non-increasing in $\Delta t$ under bounded creep: increasing the
	time available for relaxation and creep reduces the load required to attain a
	given opening. In addition,
	\begin{equation}
		\lim_{\Delta t\to 0^+} \Sigma(\Delta t,\Delta_n,\lambda_n)
		=
		\Sigma_{\mathrm{inst}}(\Delta_n,\lambda_n),
	\end{equation}
	corresponding to the instantaneous (elastic) response.
	
	If $\Sigma(\Delta t,\Delta_n,\lambda_n)$ is continuous in $\Delta t$ and admits the
	limit
	\[\Sigma_\infty(\Delta_n,\lambda_n)= \lim_{\Delta t\to\infty}\Sigma(\Delta t,\Delta_n,\lambda_n),\]
	then a sufficient condition for existence of a solution of
	Eq.~\eqref{eq:dt_equation} is
	\begin{equation}
		\Sigma_\infty(\Delta_n,\lambda_n) \le \sigma^{(\mathrm{ext})} \le
		\Sigma_{\mathrm{inst}}(\Delta_n,\lambda_n).
		\label{eq:existence_condition}
	\end{equation}
	When the inequality is strict and $\Sigma$ is strictly monotone, the solution
	$\Delta t_n$ is unique.
	
	\paragraph{Interpretation of numerical difficulties.}
	In the present context, Eq.~\eqref{eq:dt_equation} is retained as the correct
	governing relation for determining $\Delta t_n$. The main issue observed in earlier
	computations is that the numerical procedure used to solve
	Eq.~\eqref{eq:dt_equation} may fail to identify $\Delta t_n$ at some steps, even
	though a root may exist. This motivates the development of a more robust
	root-finding strategy (e.g., bracketing plus safeguarded iterations) and improved
	diagnostics for verifying the existence condition~\eqref{eq:existence_condition}
	at each step.
	
	\subsection{Properties of the continuation function $\Sigma(\Delta t)$}
	\label{sec:sigma_dt}
	
	In the incremental quasi-static formulation adopted in this work, the time
	increment $\Delta t$ is not prescribed \emph{a priori}, but is determined from
	a scalar continuation condition. For a fixed internal state at the beginning
	of an increment and a prescribed kinematic control parameter, the equilibrium
	problem yields a scalar quantity $\Sigma(\Delta t)$, which represents the
	remote load (or stress) required to satisfy equilibrium and the imposed
	constraint over the increment of duration $\Delta t$.
	
	The admissible time increment is therefore defined implicitly by the scalar
	equation
	\begin{equation}
		\Sigma(\Delta t) = \Sigma_{\mathrm{ext}},
		\label{eq:sigma_dt_root}
	\end{equation}
	where $\Sigma_{\mathrm{ext}}$ denotes the prescribed external load level.
	
	\paragraph{Dependence on $\Delta t$.}
	In the present viscoelastic formulation, the function $\Sigma(\Delta t)$
	depends on $\Delta t$ through two mechanisms. First, the effective tangent
	modulus entering the incremental stiffness operator decreases with increasing
	$\Delta t$, reflecting the transition from instantaneous to relaxed material
	response. Second, the contribution of the viscoelastic history term increases
	with $\Delta t$, providing an additional internal driving force that reduces
	the external load required to reach the prescribed kinematic state.
	Consequently, along a continuously tracked equilibrium branch,
	$\Sigma(\Delta t)$ is expected to be a decreasing function of $\Delta t$.
	
	\paragraph{Limiting behavior.}
	In the limit $\Delta t \to 0^{+}$, the viscoelastic relaxation is negligible
	and the incremental response approaches the instantaneous elastic behavior.
	In this limit, $\Sigma(\Delta t)$ attains its maximum value. Conversely, for
	large $\Delta t$, the response approaches the fully relaxed regime and
	$\Sigma(\Delta t)$ attains its minimum value. This behavior provides a natural
	basis for determining $\Delta t$ from Eq.~\eqref{eq:sigma_dt_root}.
	
	\paragraph{Existence of quasi-static increments.}
	Equation~\eqref{eq:sigma_dt_root} admits a quasi-static solution if and only if
	a root of $\Sigma(\Delta t) - \Sigma_{\mathrm{ext}} = 0$ exists for
	$\Delta t > 0$. The absence of such a root indicates that no quasi-static
	increment can satisfy the prescribed constraint at the given load level. This
	criterion provides a physically meaningful distinction between the existence
	of quasi-static solutions and purely numerical convergence issues.
	
	\paragraph{Remarks on monotonicity.}
	While the above arguments suggest a monotone decrease of $\Sigma(\Delta t)$
	with $\Delta t$, strict monotonicity is not guaranteed in the presence of
	material and geometric nonlinearities, such as cohesive softening or changes
	in the active constraint set. For this reason, the determination of $\Delta t$
	is performed using a bracketing strategy based on the sign change of
	$\Sigma(\Delta t) - \Sigma_{\mathrm{ext}}$, rather than relying solely on the
	convergence behavior of nonlinear solvers.
	
	
	
	\section{Outlook}
	
	Future work will focus on:
	\begin{itemize}
		\item reformulating the time-increment determination problem;
		\item improving robustness and existence guarantees of the auxiliary solver;
		\item reassessing numerical results after algorithmic correction.
	\end{itemize}
	
	The present document will serve as a stable reference for these developments.
\end{document}
