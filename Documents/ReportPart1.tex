%==================================================
% ПРЕАМБУЛА
%==================================================
\documentclass[12pt,a4paper]{article}

% --- Мова та кодування ---
\usepackage[ukrainian]{babel}
\usepackage[T2A]{fontenc}
\usepackage[utf8]{inputenc}

% --- Геометрія сторінки ---
\usepackage[a4paper,margin=25mm]{geometry}

% --- Математика ---
\usepackage{amsmath,amssymb,bm}

% --- Графіка ---
\usepackage{graphicx}
\usepackage{float}
\usepackage{subcaption}

% --- Таблиці та одиниці ---
\usepackage{booktabs}
\usepackage{siunitx}

% --- Посилання ---
\usepackage{hyperref}
\hypersetup{
	colorlinks=true,
	linkcolor=blue,
	citecolor=blue,
	urlcolor=blue
}

% --- Підписи ---
\usepackage{caption}
\captionsetup{
	font=small,
	labelfont=bf,
	justification=centering
}

% --- Абзаци ---
\setlength{\parindent}{1.25cm}
\setlength{\parskip}{0pt}

%==================================================
\begin{document}
\subsection*{2.3 Поширення дископодібної тріщини, залежне від часу}
\label{subsec:coin_crack}

У розділі наведено результати чисельного дослідження поширення
дископодібної тріщини, залежного від часу, у тривимірному в’язкопружному тілі.
Ріст тріщини зумовлений релаксацією напружень за сталого зовнішнього
навантаження та відбувається у квазістатичному режимі.
Аналіз спрямований на виявлення якісних особливостей еволюції
когезійної зони та напружено-деформованого стану поблизу фронту тріщини.

Розрахунки виконано з урахуванням симетрії задачі для восьмої частини
просторової області.
Поблизу фронту тріщини використано локально згущену скінченно-елементну
сітку, у межах якої введено когезійну зону, що описує процес відриву
берегів тріщини за законом «зчеплення–відрив».
Такий підхід дозволив простежити еволюцію процесної зони та
перерозподіл напружень у часі без порушення чисельної стійкості.

На рис.~\ref{fig:profiles} показано еволюцію розкриття тріщини та
когезійних сил уздовж її поверхні для послідовних моментів часу.
На початковій стадії спостерігається інкубаційний період, протягом якого
радіус тріщини залишається сталим, однак відбувається перерозподіл
напружень і локалізація деформацій у зоні фронту.
Подальше зростання тріщини супроводжується поступовим зміщенням
когезійної зони назовні, при цьому зберігається гладкий характер
розподілу як розкриття, так і когезійних сил.
Отримані результати свідчать про стабільний характер росту тріщини,
керований релаксаційними властивостями матеріалу.

\begin{figure}[H]
	\centering
	\begin{tabular}{cc}
		\includegraphics[width=0.4\textwidth]{Fig2_k000_t0.png} &
		\includegraphics[width=0.4\textwidth]{Fig2_k001_t49.3.png} \\
		\includegraphics[width=0.4\textwidth]{Fig2_k002_t88.05.png} &
		\includegraphics[width=0.4\textwidth]{Fig2_k003_t118.4.png} \\
		\includegraphics[width=0.4\textwidth]{Fig2_k004_t143.5.png} &
		\includegraphics[width=0.4\textwidth]{Fig2_k005_t164.3.png} \\
		\includegraphics[width=0.4\textwidth]{Fig2_k006_t181.7.png} &
		\includegraphics[width=0.4\textwidth]{Fig2_k007_t194.8.png} \\
		\includegraphics[width=0.4\textwidth]{Fig2_k008_t204.4.png} &
		\includegraphics[width=0.4\textwidth]{Fig2_k009_t210.6.png}
	\end{tabular}
	\caption{Еволюція розкриття тріщини та когезійних сил.}
	\label{fig:profiles}
\end{figure}

На рис.~\ref{fig:stress_field} наведено поле напружень поблизу фронту
тріщини для характерних моментів часу:
початкового моменту,
моменту завершення інкубації тріщини
та моменту, близького до ініціювання динамічного руйнування.
Видно формування кільцевої зони підвищених напружень,
яка супроводжує фронт тріщини та поступово зміщується разом із ним у
процесі росту.
Отримана картина підтверджує якісну узгодженість чисельних результатів
із уявленнями механіки руйнування та демонструє можливість
опису інкубаційних ефектів у тривимірних задачах з використанням
когезійного підходу.

\begin{figure}[H]
	\centering
	\begin{subfigure}[t]{0.32\textwidth}
		\includegraphics[width=\textwidth]{Stress0.png}
	\end{subfigure}\hfill
	\begin{subfigure}[t]{0.32\textwidth}
		\includegraphics[width=\textwidth]{Stress1.png}
	\end{subfigure}\hfill
	\begin{subfigure}[t]{0.32\textwidth}
		\includegraphics[width=\textwidth]{Stress2.png}
	\end{subfigure}
	\caption{Поле напружень поблизу фронту тріщини.}
	\label{fig:stress_field}
\end{figure}

\end{document}
%==================================================
